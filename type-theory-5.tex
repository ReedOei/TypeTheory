\documentclass[leqno,presentation,usenames,dvipsnames]{beamer}
\DeclareGraphicsExtensions{.eps,.jpg,.png,.tif}
\usepackage{amssymb, amsmath, pdfpages, amsfonts, calc, times, type1cm, latexsym, xcolor, colortbl, hyperref, bookmark}
\usepackage{mathtools}
\usepackage{bm}
\usepackage{graphicx}
\usepackage{soul}
\usepackage{tabularx}
\usepackage{multirow}
\usepackage{listings}
\usepackage{mathpartir}
\usepackage{xspace}

\usepackage[latin1]{inputenc}
\usepackage[english]{babel}

\usetheme{Szeged}
\usecolortheme{beaver}

\definecolor{websiteGreen}{RGB}{107, 224, 134}
\definecolor{silvery}{RGB}{232, 241, 248}
\definecolor{deepOrange}{RGB}{209, 126, 0}

\definecolor{softTan}{RGB}{240, 240, 223}
\definecolor{deepGreen}{RGB}{87, 149, 115}
\definecolor{lilac}{RGB}{199, 164, 202}
\definecolor{lightOlive}{RGB}{168, 166, 96}
\definecolor{deepOlive}{RGB}{101, 109, 41}

\setbeamercolor*{palette primary}{bg=websiteGreen}
\setbeamercolor*{palette secondary}{bg=white}
\setbeamercolor*{palette tertiary}{bg=silvery}
\setbeamercolor*{palette quaternary}{bg=websiteGreen}

\setbeamercolor{section in head/foot}{fg=deepOrange}
\setbeamerfont{section in head/foot}{series=\bfseries}

\setbeamercolor{titlelike}{parent=palette primary,bg=websiteGreen,fg=white}
\setbeamercolor{frametitle}{parent=palette primary,bg=websiteGreen, fg=white}

\addtobeamertemplate{frametitle}{\vspace*{-0.65\baselineskip}}{}

\newcommand{\tryS}{\textbf{\texttt{try}}\xspace}
\newcommand{\Int}{\textbf{Int}\xspace}
\newcommand{\ok}{\textbf{ok}\xspace}
\newcommand{\catchS}{\textbf{\texttt{catch}}\xspace}

\newcommand{\inl}{\textbf{\texttt{inl}}\xspace}
\newcommand{\inr}{\textbf{\texttt{inr}}\xspace}
\newcommand{\case}{\textbf{\texttt{case}}\xspace}

\newcommand{\bool}{\textbf{Bool}}
\newcommand{\fin}{\textbf{Fin}}
\newcommand{\newc}{\texttt{new}}
\newcommand{\class}{\texttt{class}}
\newcommand{\extends}{\texttt{extends}}
\newcommand{\this}{\texttt{this}}
\newcommand{\fields}{\texttt{fields}}
\newcommand{\method}{\texttt{method}}
\newcommand{\return}{\texttt{return}}
\newcommand{\K}{\mathcal{K}}
\newcommand{\Type}{\textbf{Type}}
\newcommand{\highlight}[1]{{\color{deepOrange} \textbf{#1}}}

\newtheorem*{conjecture}{Conjecture}
\newtheorem*{proposition}{Proposition}

\title{SIGPLAN}
\subtitle{Type Theory (part $5$ of $n$): Higher-kinded types}
\date{}

\input{LaTeX/macros.tex}

\begin{document}

\frame{\titlepage}

\begin{frame}{Recap}
    \begin{itemize}
        \item Last time, we talked about Featherweight Java (not very important for this week)
        \item Two times, ago we talked about System $F$---System $F_\omega$, is, naturally an extension of System $F$
    \end{itemize}
\end{frame}

\begin{frame}{Recall: Featherweight Java}
    \begin{itemize}
        \item Featherweight Java (FJ) is basically the smallest interesting subset of Java
        \item Important features: classes, methods, fields, \textbf{inheritance}
        \item But \textbf{no}: mutation, generics, built-ins (other than \texttt{Object}), access modifiers, interfaces, etc.
    \end{itemize}

\begin{tabular}{l r l l}
    \texttt{L} & \bnfdef & \texttt{class $C$ extends $C$ \{ $\overline{C~f}$; $\overline{M}$ \} } & (class def.) \\
    \texttt{M} & \bnfdef & \texttt{$C$ m($\overline{C~x}$) \{ return $e$; \} } & (method def.) \\
    \texttt{e} & \bnfdef & \texttt{$x$} \bnfalt \texttt{$e.f$} \bnfalt \texttt{$e.m(\overline{e})$} \bnfalt \texttt{new $C$($\overline{e}$)} \bnfalt \texttt{($C$)$e$} & (expressions) \\
    \texttt{P} & \bnfdef & \texttt{$\overline{\texttt{L}}$; $e$} & (programs)
\end{tabular}
\end{frame}

\begin{frame}[fragile]{Recall: System $F$}
    \begin{itemize}
        \item Recall, in System $F$ we extended the simply-typed $\lambda$-calculus with universal quantification over types
    \end{itemize}

\begin{tabular}{l r l l}
    $\tau$ & \bnfdef & $\alpha$ \bnfalt $\tau \to \tau$ \bnfalt $\forall \alpha. \tau$ \\
    E & \bnfdef & $x$ \bnfalt $\lambda x : \tau. E$ \bnfalt $E~E$ \bnfalt $\Lambda \alpha. E$ \bnfalt $E[\tau]$
\end{tabular}
\end{frame}

\begin{frame}[fragile]{And now: System $F_\omega$}
    \begin{itemize}
        \item Now we add \emph{kinds}, $\K$, which includes types\footnote{Sometimes written $*$, instead of \Type, but I think that's confusing.} and type constructors (or type operators)
        \item Kinds $\approx$ ``types for types''
        \item Having abstractions at the type level allows us to have type constructors in the language itself (think typeclasses from Haskell, or a more powerful version of interfaces/generics from Java)
        \item System $F_\omega$ is deceptively simple, so we'll spend some time showing its power
    \end{itemize}

\begin{tabular}{l r l l}
    \color{deepOrange}{$\K$} & \color{deepOrange}{\bnfdef} & \color{deepOrange}{$\Type$ \bnfalt $\K \Rightarrow \K$} \\
    $\tau$ & \bnfdef & $\alpha$ \bnfalt $\tau \to \tau$ \bnfalt \color{deepOrange}{$\forall \alpha :: \K. \tau$ \bnfalt $\lambda \alpha :: \K. \tau$ \bnfalt $\tau~\tau$} \\
    E & \bnfdef & $x$ \bnfalt $\lambda x : \tau. E$ \bnfalt $E~E$ \bnfalt $\Lambda \alpha {\color{deepOrange}{:: \K}}. E$ \bnfalt $E[\tau]$
\end{tabular}
\end{frame}

\begin{frame}[fragile]{System $F_\omega$ Example: Optional}
    \begin{itemize}
        \item Below, we define/use a \texttt{Optional} type constructor
        \item Note that we write $\lambda \alpha. \tau$ and $\forall \alpha. \tau$ instead of $\lambda \alpha :: \Type. \tau$ or $\forall \alpha :: \Type. \tau$
        \item That is, $\texttt{Optional} :: \Type \Rightarrow \Type$
    \end{itemize}
    \footnotesize
    \begin{align*}
        \texttt{Optional} = &~\lambda \alpha. \forall \beta. (\alpha \to \beta) \to \beta \to \beta \\
        \texttt{None} : &~ \forall \alpha. \texttt{Optional}~\alpha \\
        \texttt{None} = &~ \Lambda \alpha. \Lambda \beta. \lambda f : \alpha \to \beta. \lambda b : \beta. b \\
        \texttt{Some} : &~ \forall \alpha. \alpha \to \texttt{Optional}~\alpha \\
        \texttt{Some} = &~ \Lambda \alpha. \lambda a : \alpha. \Lambda \beta. \lambda f : \alpha \to \beta. \lambda b : \beta. f~a \\
        \texttt{map} : &~ \forall \alpha, \beta. (\alpha \to \beta) \to \texttt{Optional}~\alpha \to \texttt{Optional}~\beta \\
        \texttt{map} = &~ \Lambda \alpha, \beta. \lambda f : (\alpha \to \beta). \lambda x : \texttt{Optional}~\alpha. x[\texttt{Optional}~\beta]~f~\texttt{None} \\
        \texttt{isNone} : &~ \forall \alpha. \texttt{Optional}~\alpha \to \bool \\
        \texttt{isNone} = &~ ???
    \end{align*}
\end{frame}

\begin{frame}[fragile]{System $F_\omega$ Example: Optional (Cont.)}
    \small
    \begin{align*}
        \texttt{Optional} = &~\lambda \alpha. \forall \beta. (\alpha \to \beta) \to \beta \to \beta \\
        \texttt{None} : &~ \forall \alpha. \texttt{Optional}~\alpha \\
        \texttt{None} = &~ \Lambda \alpha. \Lambda \beta. \lambda f : \alpha \to \beta. \lambda b : \beta. b \\
        \texttt{Some} : &~ \forall \alpha. \alpha \to \texttt{Optional}~\alpha \\
        \texttt{Some} = &~ \Lambda \alpha. \lambda a : \alpha. \Lambda \beta. \lambda f : \alpha \to \beta. \lambda b : \beta. f~a \\
        \texttt{map} : &~ \forall \alpha, \beta. (\alpha \to \beta) \to \texttt{Optional}~\alpha \to \texttt{Optional}~\beta \\
        \texttt{map} = &~ \Lambda \alpha, \beta. \lambda f : (\alpha \to \beta). \lambda x : \texttt{Optional}~\alpha. x[\texttt{Optional}~\beta]~f~\texttt{None} \\
        \texttt{isNone} : &~ \forall \alpha. \texttt{Optional}~\alpha \to \bool \\
        \texttt{isNone} = &~ ???
    \end{align*}
\end{frame}

\begin{frame}[fragile]{System $F_\omega$ Example: Optional (Cont.)}
    \small
    \begin{align*}
        \texttt{Optional} = &~\lambda \alpha. \forall \beta. (\alpha \to \beta) \to \beta \to \beta \\
        \texttt{None} : &~ \forall \alpha. \texttt{Optional}~\alpha \\
        \texttt{None} = &~ \Lambda \alpha. \Lambda \beta. \lambda f : \alpha \to \beta. \lambda b : \beta. b \\
        \texttt{Some} : &~ \forall \alpha. \alpha \to \texttt{Optional}~\alpha \\
        \texttt{Some} = &~ \Lambda \alpha. \lambda a : \alpha. \Lambda \beta. \lambda f : \alpha \to \beta. \lambda b : \beta. f~a \\
        \texttt{map} : &~ \forall \alpha, \beta. (\alpha \to \beta) \to \texttt{Optional}~\alpha \to \texttt{Optional}~\beta \\
        \texttt{map} = &~ \Lambda \alpha, \beta. \lambda f : (\alpha \to \beta). \lambda x : \texttt{Optional}~\alpha. x[\texttt{Optional}~\beta]~f~\texttt{None} \\
        \texttt{isNone} : &~ \forall \alpha. \texttt{Optional}~\alpha \to \bool \\
        \texttt{isNone} = &~ \color{deepOrange}\Lambda \alpha. \lambda x : \alpha. x[\bool]~(\lambda y : \alpha. \false)~\true
    \end{align*}
\end{frame}

\begin{frame}[fragile]{System $F_\omega$ Evaluation}
    \begin{itemize}
        \item Evaluation is basically the same as before (the congruence rules also exist, but are omitted for space reasons)
        \item We will introduce \emph{type equivalence} soon, which is important for evaluation (and type-checking)
    \end{itemize}

    \begin{mathpar}
        \inferrule*[right=E-App]{
        }{ (\lambda x : \tau. E)~E' \to E[E'/x] }

        \inferrule*[right=E-TApp]{
        }{ (\Lambda \alpha :: \K. E)[\tau] \to E[\tau / \alpha] }
    \end{mathpar}
\end{frame}

\begin{frame}[fragile]{System $F_\omega$ Type Equivalence}
    \begin{itemize}
        \item In addition to evaluate at the expression level, we can also evaluate at the type level now (i.e., type constructors)
        \item We inductively define the equivalence relation $\equiv$\footnote{bad for actual implementations---there are better versions for that purpose that we won't discuss here.} to be the smallest equivalence relation satisfying the following rules:
         % \item The most important rule is \textsc{Eq-App}, as it ``implements'' type constructors
    \end{itemize}

    \small
    \begin{mathpar}
        \inferrule*[right=Eq-App]{
        }{ (\lambda \alpha :: \K. \tau)~\sigma \equiv \tau[\sigma / \alpha] }

        \inferrule*[right=Eq-All]{
            \tau \equiv \tau'
        }{ \forall \alpha :: \K. \tau \equiv \forall \alpha :: \K. \tau' }

        \inferrule*[right=Eq-Abs]{
            \tau \equiv \tau'
        }{ \lambda \alpha :: \K. \tau \equiv \lambda \alpha :: \K. \tau' }

        \inferrule*[right=Eq-App]{
            \tau \equiv \tau'
            \\
            \sigma \equiv \sigma'
        }{ \tau~\sigma \equiv \tau'~\sigma' }

        \inferrule*[right=Eq-Fun]{
            \tau \equiv \tau'
            \\
            \sigma \equiv \sigma'
        }{ \tau \to \sigma \equiv \tau' \to \sigma' }
    \end{mathpar}
\end{frame}

\begin{frame}[fragile]{System $F_\omega$ Kind Rules}
    \begin{itemize}
        \item In addition to typechecking, we also need to ``kind-check'', or ``typecheck the types''
        \item This ensures our type-level computations will not go awry
        \item We extend type environments to be able to contain type variables and their kind (e.g., we can have $\alpha :: \Type \in \Gamma$)
        \item First, the rules for type-level computations
    \end{itemize}

    \begin{mathpar}
        \inferrule*[right=K-Var]{
        }{ \Gamma, \alpha :: \K \proves \alpha :: K }

        \inferrule*[right=K-Abs]{
            \Gamma, \alpha :: \K \proves \tau :: \K'
        }{ \Gamma \proves (\lambda \alpha :: \K. \tau) :: \K \Rightarrow \K' }

        \inferrule*[right=K-App]{
            \Gamma \proves \tau :: \K \Rightarrow \K'
            \\
            \Gamma \proves \sigma :: \K
        }{ \Gamma \proves \tau~\sigma :: \K' }
    \end{mathpar}
\end{frame}

\begin{frame}[fragile]{System $F_\omega$ Kind Rules}
    \begin{itemize}
        \item Now, the rules for checking regular types
        \item We basically need to make sure that certain things evaluate to expressions (but the previous rules control things that can evaluate to types)
    \end{itemize}

    \begin{mathpar}
        \inferrule*[right=K-Fun]{
            \Gamma \proves \tau :: \Type
            \\
            \Gamma \proves \sigma :: \Type
        }{ \Gamma \proves (\tau \to \sigma) :: \Type }

        \inferrule*[right=K-All]{
            \Gamma, \alpha :: \K \proves \tau :: \Type
        }{ \Gamma \proves (\forall \alpha :: \K. \tau) :: \Type }
    \end{mathpar}
\end{frame}

\begin{frame}[fragile]{System $F_\omega$ Type Rules}
    \begin{itemize}
        \item The type rules are largely the same, with the caveat we now need to kind-check all the types
        \item Note: no type constructors appear in these rules!
    \end{itemize}

    \small
    \begin{mathpar}
        \inferrule*[right=T-Abs]{
            \Gamma \proves \tau :: \Type
            \\
            \Gamma, x : \tau \proves E : \sigma
        }{ \Gamma \proves (\lambda x : \tau. E) : \tau \to \sigma }

        \inferrule*[right=T-TAbs]{
            \Gamma, \alpha :: \K \proves E : \tau
        }{ \Gamma \proves (\Lambda \alpha :: \K. E) : \forall \alpha :: \K. \tau }

        \inferrule*[right=T-TApp]{
            \Gamma \proves E : \forall \alpha :: \K. \sigma
            \\
            \Gamma \proves \tau :: \K
        }{ \Gamma \proves E~[\tau] : \sigma[\tau / \alpha] }

        \inferrule*[right=T-Eq]{
            \Gamma \proves E : \tau
            \\
            \tau \equiv \sigma
            \\
            \Gamma \proves \sigma :: \Type
        }{ \Gamma \proves E : \sigma }
    \end{mathpar}
\end{frame}

% \begin{frame}[fragile]{Typechecking \texttt{length}}
%     \Wider{%
%     \tiny
%     Let $\Gamma = \{ c :: \Type \Rightarrow \Type, \alpha \}$.

%     \begin{mathpar}
%         \inferrule*{
%             \inferrule*{
%                 \inferrule*{
%                     \inferrule*{
%                     }{ \Gamma \proves \texttt{Container}~c~\alpha :: \Type }
%                     \\
%                     \inferrule*{
%                     }{ \Gamma, xs : \texttt{Container}~c~\alpha \proves xs[\N]~(\lambda x : \alpha, y : \N. y + 1)~0 : \N }
%                 }{ \Gamma \proves \lambda xs : \texttt{Container}~c~\alpha. xs[\N]~(\lambda x : \alpha, y : \N. y + 1)~0) : \texttt{Container}~c~\alpha \to \N }
%             } { c :: \Type \Rightarrow \Type \proves (\Lambda \alpha. \lambda xs : \texttt{Container}~c~\alpha. xs[\N]~(\lambda x : \alpha, y : \N. y + 1)~0) : \forall \alpha. \texttt{Container}~c~\alpha \to \N }
%         }{ \emptyset \proves \texttt{length} : \forall c :: \Type \Rightarrow \Type. \forall \alpha. \texttt{Container}~c~\alpha \to \N }
%             % \texttt{length} = & \Lambda c :: \Type \Rightarrow \Type. \Lambda \alpha. \\
%             %                   & \lambda xs : \texttt{Container}~c~\alpha. xs[\N]~(\lambda x : \alpha, y : \N. y + 1)~0
%     \end{mathpar}}
% \end{frame}

\begin{frame}[fragile]{Another System $F_\omega$ Example: Foldable}
    \begin{itemize}
        \item Below, we define/use a \texttt{Foldable} type, for containers that can be ``summed''
    \end{itemize}

    \footnotesize
    \begin{align*}
        \texttt{Foldable} = & \lambda f :: \Type \Rightarrow \Type. \forall \alpha, \beta. (\alpha \to \beta \to \beta) \to \beta \to f~\alpha \to \beta \\
        \texttt{length} : & \forall f :: \Type \Rightarrow \Type. \forall \alpha. \texttt{Foldable}~f~\alpha \to \N \\
        \texttt{length} = & \Lambda f :: \Type \Rightarrow \Type. \Lambda \alpha. \\
                          & \lambda xs : \texttt{Foldable}~f~\alpha. xs[\N]~(\lambda x : \alpha, y : \N. y + 1)~0
    \end{align*}

\begin{lstlisting}[language=haskell, basicstyle=\small\ttfamily, mathescape=true]
length :: Foldable f => f a => Nat
length xs = foldr (\a b -> b + 1) 0 xs
\end{lstlisting}
\end{frame}

\begin{frame}[fragile]{Another System $F_\omega$ Example: Container}
    \begin{itemize}
        \item Here we have a container\footnote{This \texttt{Container} is actually a monad, but shhh...}
        \item It has three functions:
            \begin{itemize}
                \item $\texttt{singleton} : \forall \alpha. \alpha \to c~\alpha$
                \item $\texttt{flatten} : \forall \alpha. c~(c~\alpha) \to c~\alpha$
                \item $\texttt{map} : \forall \alpha,\beta. (\alpha \to \beta) \to c~\alpha \to c~\beta) \times$
            \end{itemize}
    \end{itemize}
    \footnotesize
    \begin{align*}
        \texttt{Container} = &~\lambda c :: \Type \Rightarrow \Type. \\
                             &~\qquad(\forall \alpha. \alpha \to c~\alpha) \times (\forall \alpha. c~(c~\alpha) \to c~\alpha) \times \\
                             &~\qquad(\forall \alpha,\beta. (\alpha \to \beta) \to c~\alpha \to c~\beta) \\
        \texttt{flatMap} :&~ \forall c :: \Type \Rightarrow \Type. \forall \alpha, \beta. \\
                          &~\qquad \texttt{Container}~c~\alpha \to c~\alpha \to (\alpha \to c~\beta) \to c~\beta \\
        \texttt{flatMap} =&~ \Lambda c :: \Type \Rightarrow \Type. \Lambda \alpha, \beta. \\
                          &~\qquad \lambda \texttt{impl} : \texttt{Container}~c~\alpha. \lambda xs : c~\alpha. \lambda f : \alpha \to c~\beta. \\
                          &~\qquad \texttt{impl.flatten}[\alpha]~(\texttt{impl.map}[\alpha,c~\beta]~f~xs)
    \end{align*}
\end{frame}

\begin{frame}[fragile]{Another System $F_\omega$ Example: Monad (Haskell Names)}
    \begin{itemize}
        \item Here we have a container\footnote{This \texttt{Container} is actually a monad, but shhh...}
        \item It has three functions:
            \begin{itemize}
                \item $\highlight{pure} : \forall \alpha. \alpha \to c~\alpha$
                \item $\highlight{join} : \forall \alpha. c~(c~\alpha) \to c~\alpha$
                \item $\highlight{fmap} : \forall \alpha,\beta. (\alpha \to \beta) \to c~\alpha \to c~\beta) \times$
            \end{itemize}
    \end{itemize}
    \footnotesize
    \begin{align*}
        \highlight{Monad} = &~\lambda c :: \Type \Rightarrow \Type. \\
                             &~\qquad(\forall \alpha. \alpha \to c~\alpha) \times (\forall \alpha. c~(c~\alpha) \to c~\alpha) \times \\
                             &~\qquad(\forall \alpha,\beta. (\alpha \to \beta) \to c~\alpha \to c~\beta) \\
        (\highlight{$>>=$}) :&~ \forall c :: \Type \Rightarrow \Type. \forall \alpha, \beta. \\
                          &~\qquad \texttt{Container}~c~\alpha \to c~\alpha \to (\alpha \to c~\beta) \to c~\beta \\
        (\highlight{$>>=$}) =&~ \Lambda c :: \Type \Rightarrow \Type. \Lambda \alpha, \beta. \\
                          &~\qquad \lambda \texttt{impl} : \texttt{Monad}~c~\alpha. \lambda xs : c~\alpha. \lambda f : \alpha \to c~\beta. \\
                          &~\qquad \texttt{impl.join}[\alpha]~(\texttt{impl.fmap}[\alpha,c~\beta]~f~xs)
    \end{align*}
\end{frame}

\begin{frame}[fragile]{Next Steps}
    \begin{itemize}
        \item We'll continue meeting next semester, but potentially at a different time (or maybe not).
            Also, I'll be graduating soon, so if someone is interested in taking over let me know (pls)
        \item Think about things you want to hear about for next semester
        \item Possible topics we have:
            \begin{itemize}
                \item Some type theory topics:
                    \begin{itemize}
                        \item Dependent types (like in Coq, Agda)
                        \item Pure type systems (unifies the various $\lambda$-calculi)
                        \item Recursive types
                        \item Bounded polymorphism
                        \item Whatever else I think is cool
                    \end{itemize}
                \item Something not type theory: compilers, interpreters, program verification\footnote{not PL, per se, but I don't mind}, proof assistants
            \end{itemize}
    \end{itemize}
\end{frame}

\begin{frame}{Empty Slide for Scratch Work}
\end{frame}

\begin{frame}{Existentials}
    \footnotesize
    \begin{align*}
        \texttt{EitherSig} = & \lambda \alpha,\beta. \forall \gamma. (\forall \texttt{Either} :: \Type \Rightarrow \Type \Rightarrow \Type. \\
                             & \qquad (\texttt{Either}~\alpha~\beta \to \alpha) \to (\texttt{Either}~\alpha~\beta \to \beta) \to \gamma) \to \gamma \\
    \end{align*}
\end{frame}

\end{document}

