\documentclass[leqno,presentation,usenames,dvipsnames]{beamer}
\DeclareGraphicsExtensions{.eps,.jpg,.png,.tif}
\usepackage{amssymb, amsmath, pdfpages, amsfonts, calc, times, type1cm, latexsym, xcolor, colortbl, hyperref, bookmark}
\usepackage{mathtools}
\usepackage{graphicx}
\usepackage{tabularx}
\usepackage{multirow}
\usepackage{listings}
\usepackage{mathpartir}
\usepackage{xspace}

\usepackage[latin1]{inputenc}
\usepackage[english]{babel}

\usetheme{Szeged}
\usecolortheme{beaver}

\definecolor{websiteGreen}{RGB}{107, 224, 134}
\definecolor{silvery}{RGB}{232, 241, 248}
\definecolor{deepOrange}{RGB}{209, 126, 0}

\definecolor{softTan}{RGB}{240, 240, 223}
\definecolor{deepGreen}{RGB}{87, 149, 115}
\definecolor{lilac}{RGB}{199, 164, 202}
\definecolor{lightOlive}{RGB}{168, 166, 96}
\definecolor{deepOlive}{RGB}{101, 109, 41}

\setbeamercolor*{palette primary}{bg=websiteGreen}
\setbeamercolor*{palette secondary}{bg=white}
\setbeamercolor*{palette tertiary}{bg=silvery}
\setbeamercolor*{palette quaternary}{bg=websiteGreen}

\setbeamercolor{section in head/foot}{fg=deepOrange}
\setbeamerfont{section in head/foot}{series=\bfseries}

\setbeamercolor{titlelike}{parent=palette primary,bg=websiteGreen,fg=white}
\setbeamercolor{frametitle}{parent=palette primary,bg=websiteGreen, fg=white}

\addtobeamertemplate{frametitle}{\vspace*{-0.65\baselineskip}}{}

\newcommand{\tryS}{\textbf{\texttt{try}}\xspace}
\newcommand{\Int}{\textbf{Int}\xspace}
\newcommand{\ok}{\textbf{ok}\xspace}
\newcommand{\catchS}{\textbf{\texttt{catch}}\xspace}
\newcommand{\overbar}[1]{\mkern1.5mu\overline{\mkern-1.5mu#1\mkern-1.5mu}\mkern1.5mu}
\newcommand{\highlight}[1]{
  \addtolength{\fboxrule}{.2ex}
  \begin{block}{}
    \begin{quote}#1
    \end{quote}
  \end{block}
}

\newtheorem*{conjecture}{Conjecture}
\newtheorem*{proposition}{Proposition}

\title{SIGPLAN}
\subtitle{Type Theory (part $1$ of $n$): $\lambda$-calculus}
\date{}

\input{LaTeX/macros.tex}

\begin{document}

\frame{\titlepage}

\begin{frame}{Why Type Theory?: History}
    \begin{itemize}
        \item Historically: type theory was created to resolve issues in the foundations of mathematics
        \item Russel's Paradox:
            \[
                A = \setbuild{x}{x \not\in x}
            \]

            If $A \in A$, then $A \not\in A$.
            If $A \not\in A$, then $A \in A$.
            Whoops.

        \item Naive encoding with types:
            \[
                A : \texttt{set}~(\texttt{set}~T) = \setbuild{x : \texttt{set}~T}{x \not\in x}
            \]

            $A$ does not have type $\texttt{set}~T$, so there's no problem.\footnote{This is also what the axiom schema of separation (also known as restricted comprehension)) does in ZFC}
    \end{itemize}
\end{frame}

\begin{frame}{Why Type Theory?: Today}
    \begin{itemize}
        \item Type theory is still being studied as an alternate foundation for mathematics (HoTT).
        \item But for now: turns out, type theory and its ideas are very useful in programming languages
        \item We can show that our languages are type safe/satisfy other nice properties (e.g., programs terminate)
        \item Also, it's fun(ctional)
    \end{itemize}
\end{frame}

\begin{frame}{Outline}
    \begin{itemize}
        \item \textbf{The Untyped $\lambda$-calculus}
        \item Untyped $\lambda$-calclus: Evaluation
        \item Simple-typed $\lambda$-calculus
        \item Simple-typed $\lambda$-calculus: Implementation
        \item Simple-typed $\lambda$-calculus: Proofs
    \end{itemize}
\end{frame}

\begin{frame}{The Untyped $\lambda$-Calculus}
    \begin{itemize}
        \item The simplest programming language is the (untyped) lambda calculus
        \item It is Turing-complete (can compute anything a regular language can)
        \item For our purposes, it provides a simple language to let us study the ideas of type theory---it is quite extendable
        \item Programs consist of only expressions, of which there are three kinds\footnote{Pedants might insist I include parenthesized expressions.}: \emph{variables}, \emph{abstractions} (also called $\lambda$ functions), and \emph{applications}.
    \end{itemize}

\begin{tabular}{l r l l}
    E & \bnfdef & $x$ & \bnfalt $\lambda x. E$ \bnfalt $E~E$
\end{tabular}
\end{frame}

\begin{frame}{$\lambda$-Calculus: Examples}
    \begin{itemize}
        \item We typically think of $\lambda x. E$ as a function like $f(x) = E$ (or $x \mapsto E$)
        \item So $(\lambda x. E)~E'$ ``means'' $f(E')$, where $f(x) = E$.
        \item We read $E \to E'$ as $E$ \emph{reduces/evaluates/simplifies to} $E'$; it captures the notion of computation.
    \end{itemize}

    \small
\begin{align*}
    &(\lambda x. x)~y \to y \\
    &((\lambda x. \lambda y. y)~a)~b \to (\lambda y. y)~b \to b \\
    &(((\lambda x. \lambda y. y)~a)~(\lambda x. x))~b \to
     ((\lambda y. y)~(\lambda x. x))~b \to
     (\lambda x. x)~b \to b \\
\end{align*}
\end{frame}

\begin{frame}{$\lambda$-Calculus: Some toy programs}
    \begin{itemize}
        \item The untyped $\lambda$-calculus is Turing-complete: we will show only a small subset of what you can do
        \item Can encode everything: recursion, numbers, booleans, lists, etc.
        \item Church encoding of booleans and common operations:
    \end{itemize}
\begin{align*}
    \texttt{True} & := (\lambda x. \lambda y. x) \\
    \texttt{False} & := (\lambda x. \lambda y. y) \\
    \texttt{And} & := (\lambda b_1. \lambda b_2. b_1~b_2~\texttt{False}) \\
    \texttt{Not} & := (\lambda b. b~\texttt{False}~\texttt{True}) \\
    \texttt{Or} & := ?????
\end{align*}
\end{frame}

\begin{frame}{Outline}
    \begin{itemize}
        \item The Untyped $\lambda$-calculus
        \item \textbf{Untyped $\lambda$-calclus: Evaluation}
        \item Simple-typed $\lambda$-calculus
        \item Simple-typed $\lambda$-calculus: Implementation
        \item Simple-typed $\lambda$-calculus: Proofs
    \end{itemize}
\end{frame}

\begin{frame}{$\lambda$-Calculus: Evaluating ``\texttt{And True False}''}
\begin{align*}
    \texttt{And}~\texttt{True}~\texttt{False} & = (\lambda b_1. \lambda b_2. b_1~b_2~\texttt{False}) (\lambda x. \lambda y. x) (\lambda z. \lambda w. w) \\
                                        & \to (\lambda b_2. (\lambda x. \lambda y. x)~b_2~\texttt{False}) (\lambda z. \lambda w. w) \\
                                        & \to ((\lambda x. \lambda y. x)~(\lambda z. \lambda w. w)~\texttt{False}) \\
                                        & \to ((\lambda y. (\lambda z. \lambda w. w))~\texttt{False}) \\
                                        & \to (\lambda z. \lambda w. w) \\
                                        & = \texttt{False}
\end{align*}
\end{frame}

\begin{frame}{$\lambda$-Calculus: Evaluation}
    \begin{itemize}
        \item We evaluate programs according to the three rules below (congruence, $\beta$-reduction, and $\alpha$-conversion)
        \item $E[E'/x]$ means \emph{replace $x$ by $E'$ in $E$}; for example, $(x y x)[(\lambda z. z)/x] = (\lambda z. z) y (\lambda z. z)$.
        \item Sometimes written $E[x \mapsto E']$, $E[x/E']$, $E[x := E']$, etc.
    \end{itemize}

    \begin{align*}
        E_1~E_2              & \to E_1'~E_2 & \tif E_1 \to E_1' \\
        (\lambda x. E_1) E_2 &\to_{\beta} E_1[E_2/x] \\
        (\lambda x. E)       &\to_{\alpha} (\lambda y. E[y/x]) & \tif \text{$y$ is not free in $E$}
    \end{align*}
\end{frame}

\begin{frame}{$\lambda$-Calculus: Evaluation (Inference rules)}
    \begin{itemize}
        \item These are usually presented via \emph{inference rules} (if everything on the top (\emph{premises}) is true, we can conclude the bottom (\emph{conclusion})).
        \item We sometimes omit the line when there are no premises
        \item The ``if $y$ is not free in $E$'' part on \textsc{$\alpha$-conv} is call a \emph{side-condition}.
    \end{itemize}

\begin{mathpar}
    \inferrule*[right=Congr.]{
        E_1 \to E_1'
    }{ E_1~E_2 \to E_1'~E_2 }

    \inferrule*[right=$\beta$-reduce]{
    }{ (\lambda x. E) E' \to E[E'/x] }

    \inferrule*[right=$\beta$-reduce-alt.]{}{ (\lambda x. E) E' \to E[E'/x] }

    \inferrule*[right=$\alpha$-conv.]{
    }{ (\lambda x. E) \to (\lambda y. E[y/x]) }
    \tif \text{$y$ is not free in $E$}
\end{mathpar}
\end{frame}

\begin{frame}{Outline}
    \begin{itemize}
        \item The Untyped $\lambda$-calculus
        \item Untyped $\lambda$-calclus: Evaluation
        \item \textbf{Simple-typed $\lambda$-calculus}
        \item Simple-typed $\lambda$-calculus: Implementation
        \item Simple-typed $\lambda$-calculus: Proofs
    \end{itemize}
\end{frame}

\begin{frame}{Simply-Typed $\lambda$-Calculus}
    \begin{itemize}
        \item We've made it $X$ minutes into ``type theory'' without talking about types in the $\lambda$-calculus
        \item There are many type systems we can put on top of the untyped $\lambda$-calculus
        \item The simplest (we need \texttt{bool} or something to have any well-formed types):
    \end{itemize}

\begin{tabular}{l r l l}
    $\tau$ & \bnfdef & \texttt{bool} \bnfalt $\tau \to \tau$ \\
    E & \bnfdef & $\true$ \bnfalt $\false$ \bnfalt $x$ \bnfalt $\lambda x : \tau. E$ \bnfalt $E~E$
\end{tabular}
\end{frame}

\begin{frame}{The Type Rules}
    \begin{itemize}
        \item We write $x : \tau$ to mean ``$x$ has type $\tau$''.
        \item We write $\Gamma \proves E : \tau$, read ``in the environment $\Gamma$, $E$ has type $\tau$''
        \item $\Gamma$ is a \emph{type environment}: it's a list of pairs $x_1 : \tau_1, x_2 : \tau_2, \ldots, x_n : \tau_n$. \footnote{doesn't technically have to be finite, I guess}
        \item In our simple language, we give one type rule for each sort of expression
        \item The identity function on booleans:
    \end{itemize}
    \begin{align*}
        (\lambda x : \texttt{bool}. x)
    \end{align*}
\end{frame}

\begin{frame}{Checking Boolean Constants}
    \begin{itemize}
        \item We added boolean constnats, so we also need rules for them
        \item Multiple ways to write this (the second is somewhat nicer, because there's fewer rules)
    \end{itemize}

\begin{mathpar}
    \inferrule*[right=True]{
    }{ \Gamma \proves \true : \texttt{bool} }

    \inferrule*[right=False]{
    }{ \Gamma \proves \false : \texttt{bool} }

    \inferrule*[right=Bool]{
        b \in \curlys{\true, \false}
    }{ \Gamma \proves b : \texttt{bool} }
\end{mathpar}
\end{frame}

\begin{frame}{The \textsc{Var} Rule}
    \begin{itemize}
        \item To know what type a variable is, it must be in the type environment
        \item Then we simply use whatever type the environment says the variable has
        \item This let's us reject programs that don't declare all their variables
    \end{itemize}

\begin{mathpar}
    \inferrule*[right=Var]{
    }{ \Gamma, x : \tau \proves x : \tau }
\end{mathpar}
\end{frame}

\begin{frame}{The \textsc{App} Rule}
    \begin{itemize}
        \item The \textsc{App} rule ensures that we only apply functions to arguments (e.g., we don't do $\texttt{true}(x)$).
    \end{itemize}

\begin{mathpar}
    \inferrule*[right=App]{
        \Gamma \proves f : \tau_1 \to \tau_2
        \\
        \Gamma \proves x : \tau_1
    }{ \Gamma \proves f~x : \tau_2 }
\end{mathpar}
\end{frame}

\begin{frame}{The \textsc{Abs} Rule}
    \begin{itemize}
        \item The \textsc{Abs} rule ensures $\lambda$-abstractions are well-typed: the \emph{body} can use the variable $x$ as though it were of type $\tau$
    \end{itemize}

\begin{mathpar}
    \inferrule*[right=Abs]{
        \Gamma, x : \tau_1 \proves E : \tau_2
    }{ \Gamma \proves (\lambda x : \tau_1. E) : \tau_1 \to \tau_2 }
\end{mathpar}
\end{frame}

\begin{frame}{Typechecking a Program}
    \begin{itemize}
        \item Below is a \emph{proof tree}, showing that a program is \emph{well-typed}\footnote{it's hard to fit anything bigger on one slide}
    \end{itemize}

\begin{minipage}{\dimexpr\textwidth+3em\relax}
\small
\begin{mathpar}
    \inferrule*[right=App]{
        \inferrule*[right=Abs]{
            \inferrule*[right=Var]{
            }{ x : \texttt{bool} \proves x : \texttt{bool} }
        }{ \emptyset \proves (\lambda x : \texttt{bool}. \true) : \texttt{bool} \to \texttt{bool} }
        \\
        \inferrule*[right=Bool]{
        }{ \emptyset \proves \false : \texttt{bool} }
    }{ \emptyset \proves (\lambda x : \texttt{bool}. \true)~\false : \texttt{bool} }
\end{mathpar}
\end{minipage}%
\end{frame}

\begin{frame}{Benefits of Formalization}
    \begin{itemize}
        \item We can now prove theorems about the language
        \item There is now an unambiguous definition what the language is and what it does: therefore, we can say whether an implementation is correct or not.
        \item Helps with the design process a lot!
    \end{itemize}
\end{frame}

\begin{frame}{Notes about simply-typed $\lambda$-calculus}
    \begin{itemize}
        \item The simply-typed $\lambda$-calculus is \textbf{weaker} (much weaker) than the untyped version
        \item We reject many valid (but also many invalid!) programs
        \item To get more valid programs, we need a more complicated type system
        \item In exchange, we get nice guarantees about our programs (i.e., strong normalization)
    \end{itemize}
\end{frame}

\begin{frame}{Outline}
    \begin{itemize}
        \item The Untyped $\lambda$-calculus
        \item Untyped $\lambda$-calclus: Evaluation
        \item Simple-typed $\lambda$-calculus
        \item \textbf{Simple-typed $\lambda$-calculus: Implementation}
        \item Simple-typed $\lambda$-calculus: Proofs
    \end{itemize}
\end{frame}

\begin{frame}[fragile]{Implementation of a typechecker}
    \begin{itemize}
        \item Let's implement these rules in Haskell.
        \item First we define types representing typons (\texttt{Ty}) and expressions (\texttt{Expr}) in our language
        \item Then we define a function \texttt{check}, such that $\Gamma \proves e : \tau$ if and only if \lstinline[language=haskell, basicstyle=\small\ttfamily]{check gamma e = Just tau}
        \item \emph{Ill-typed} expressions (e.g., $\lambda x : \texttt{bool} . y$) will give us \lstinline[language=haskell, basicstyle=\small\ttfamily]{check gamma e = Nothing}
    \end{itemize}
\begin{lstlisting}[language=haskell, basicstyle=\small\ttfamily]
data Ty = B | Func Ty Ty
data Expr = T | F | Var String
          | App Expr Expr | Abs String Ty Expr
type Env = String -> Maybe Ty
check :: Env -> Expr -> Maybe Ty
\end{lstlisting}
\end{frame}

\begin{frame}[fragile]{Implementation: Constants and Variables}
    \begin{itemize}
        \item First the \textsc{Bool} and \textsc{Var} rules:
    \end{itemize}
\begin{lstlisting}[language=haskell, basicstyle=\small\ttfamily]
check gamma T = Just bool
check gamma F = Just bool
check gamma (Var x) = gamma x
\end{lstlisting}
\end{frame}

\begin{frame}[fragile]{Implementation: Functions}
    \begin{itemize}
        \item Now for the \textsc{App} and \textsc{Abs} rules:
    \end{itemize}
\begin{lstlisting}[language=haskell, basicstyle=\small\ttfamily]
check gamma (App f x) =
    case (check gamma f, check gamma x) of
        (Just (Func t1 t2), Just t3) ->
            if t1 == t3 then Just t2
            else Nothing
        _ -> Nothing
check gamma (Abs x t body) =
  let newGamma = (\y -> if x == y then Just t
                        else gamma y)
  in case check newGamma body of
      Just t2 -> Just (Func t t2)
      Nothing -> Nothing
\end{lstlisting}
\end{frame}

\begin{frame}{Outline}
    \begin{itemize}
        \item The Untyped $\lambda$-calculus
        \item Untyped $\lambda$-calclus: Evaluation
        \item Simple-typed $\lambda$-calculus
        \item Simple-typed $\lambda$-calculus: Implementation
        \item \textbf{Simple-typed $\lambda$-calculus: Proofs}
    \end{itemize}
\end{frame}

\begin{frame}{Proving Type Safety}
    \begin{itemize}
        \item Typically we split proving type safety into two parts: \emph{progress} and \emph{preservation}
        \item Informally, type safety means ``well-typed programs don't get stuck.''
    \end{itemize}

\begin{theorem}[Progress]
    Let $E$ be a closed expression (i.e., no free variables).
    If $\Gamma \proves E : \tau$, then either $E$ is a value (i.e, $E = \true$, $E = \false$, or $E = \lambda x. E$), or there is some $E'$ such that $E \to E'$.
\end{theorem}

\begin{theorem}[Preservation]
    If $\Gamma \proves E : \tau$ and $E \to E'$, then $\Gamma \proves E' : \tau$.\footnote{Here we use the same $\tau$ for both; in some languages this may not hold (e.g., because of subtyping)}
\end{theorem}
\end{frame}

\end{document}

