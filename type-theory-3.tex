\documentclass[leqno,presentation,usenames,dvipsnames]{beamer}
\DeclareGraphicsExtensions{.eps,.jpg,.png,.tif}
\usepackage{amssymb, amsmath, pdfpages, amsfonts, calc, times, type1cm, latexsym, xcolor, colortbl, hyperref, bookmark}
\usepackage{mathtools}
\usepackage{bm}
\usepackage{graphicx}
\usepackage{soul}
\usepackage{tabularx}
\usepackage{multirow}
\usepackage{listings}
\usepackage{mathpartir}
\usepackage{xspace}

\usepackage[latin1]{inputenc}
\usepackage[english]{babel}

\usetheme{Szeged}
\usecolortheme{beaver}

\definecolor{websiteGreen}{RGB}{107, 224, 134}
\definecolor{silvery}{RGB}{232, 241, 248}
\definecolor{deepOrange}{RGB}{209, 126, 0}

\definecolor{softTan}{RGB}{240, 240, 223}
\definecolor{deepGreen}{RGB}{87, 149, 115}
\definecolor{lilac}{RGB}{199, 164, 202}
\definecolor{lightOlive}{RGB}{168, 166, 96}
\definecolor{deepOlive}{RGB}{101, 109, 41}

\setbeamercolor*{palette primary}{bg=websiteGreen}
\setbeamercolor*{palette secondary}{bg=white}
\setbeamercolor*{palette tertiary}{bg=silvery}
\setbeamercolor*{palette quaternary}{bg=websiteGreen}

\setbeamercolor{section in head/foot}{fg=deepOrange}
\setbeamerfont{section in head/foot}{series=\bfseries}

\setbeamercolor{titlelike}{parent=palette primary,bg=websiteGreen,fg=white}
\setbeamercolor{frametitle}{parent=palette primary,bg=websiteGreen, fg=white}

\addtobeamertemplate{frametitle}{\vspace*{-0.65\baselineskip}}{}

\newcommand{\tryS}{\textbf{\texttt{try}}\xspace}
\newcommand{\Int}{\textbf{Int}\xspace}
\newcommand{\ok}{\textbf{ok}\xspace}
\newcommand{\catchS}{\textbf{\texttt{catch}}\xspace}

\newcommand{\inl}{\textbf{\texttt{inl}}\xspace}
\newcommand{\inr}{\textbf{\texttt{inr}}\xspace}
\newcommand{\case}{\textbf{\texttt{case}}\xspace}

\newcommand{\bool}{\textbf{Bool}}
\newcommand{\fin}{\textbf{Fin}}
\newcommand{\newc}{\texttt{new}}
\newcommand{\class}{\texttt{class}}
\newcommand{\extends}{\texttt{extends}}
\newcommand{\this}{\texttt{this}}
\newcommand{\fields}{\texttt{fields}}
\newcommand{\method}{\texttt{method}}
\newcommand{\return}{\texttt{return}}

\newtheorem*{conjecture}{Conjecture}
\newtheorem*{proposition}{Proposition}

\title{SIGPLAN}
\subtitle{Type Theory (part $3$ of $n$): System $F$, Church Encodings}
\date{}

\input{LaTeX/macros.tex}

\begin{document}

\frame{\titlepage}

\begin{frame}{Recap}
    \begin{itemize}
        \item Last time, we introduced talked about extensions to the simply-typed $\lambda$-calculus, algebraic data types, and progress/preservation
        \item TODO: ADD MORE INTERACTION?
        \item Add outline
    \end{itemize}
\end{frame}

\begin{frame}{Recap: Product Types}
    \begin{tabular}{l r l l}
        $\tau$ & \bnfdef & $\ldots$ \bnfalt $\tau \times \tau$ \\
        E & \bnfdef & $\ldots$ \bnfalt $(E,E)$ \bnfalt $E_\ell$ \bnfalt $E_r$
    \end{tabular}

    \begin{mathpar}
        \inferrule*[right=Pair-Proj-L]{
        }{ (E_1, E_2)_\ell \to E_1 }

        \inferrule*[right=Pair-Proj-R]{
        }{ (E_1, E_2)_r \to E_1 }
    \end{mathpar}

    \begin{mathpar}
        \inferrule*[right=T-Pair]{
            \Gamma \proves E_1 : \tau_1
            \\
            \Gamma \proves E_2 : \tau_2
        }{ \Gamma \proves (E_1, E_2) : \tau_1 \times \tau_2 }

        \inferrule*[right=T-Proj-L]{
            \Gamma \proves E : \tau_1 \times \tau_2
        }{ \Gamma \proves E_\ell : \tau_1 }

        \inferrule*[right=T-Proj-R]{
            \Gamma \proves E : \tau_1 \times \tau_2
        }{ \Gamma \proves E_r : \tau_2 }
    \end{mathpar}
\end{frame}

\begin{frame}{Recap: Sum Types}
\begin{tabular}{l r l l}
    $\tau$ & \bnfdef & $\ldots$ \bnfalt $\tau + \tau$ \\
    E & \bnfdef & $\ldots$ \bnfalt $\inl~E$ \bnfalt $\inr~E$ \bnfalt $\case(E)~\inl~x \mapsto E ; \inr~x \mapsto E$
\end{tabular}

\begin{mathpar}
    \inferrule*[right=Case-Congr]{
        E \to E'
    }{ (\case(E)~\inl~x \mapsto E_1 ; \inr~y \mapsto E_2) \\\to (\case(E')~\inl~x \mapsto E_1 ; \inr~y \mapsto E_2) }

    \inferrule*[right=Case-L]{
    }{ (\case(\inl~E)~\inl~x \mapsto E_1 ; \inr~y \mapsto E_2) \to E_1[E/x] }

    \inferrule*[right=Case-R]{
    }{ (\case(\inr~E)~\inl~x \mapsto E_1 ; \inr~y \mapsto E_2) \to E_2[E/y] }
\end{mathpar}
\end{frame}

\begin{frame}{Recap: Sum Types (Typing)}
\begin{mathpar}
    \inferrule*[right=T-In-L]{
        \Gamma \proves E : \tau_1
    }{ \Gamma \proves \inl~E : \tau_1 + \tau_2 }

    \inferrule*[right=T-In-R]{
        \Gamma \proves E : \tau_2
    }{ \Gamma \proves \inr~E : \tau_1 + \tau_2 }

    \inferrule*[right=T-Case]{
        \Gamma \proves E : \tau_1 + \tau_2
        \\
        \Gamma, x : \tau_1 \proves E_1 : \tau
        \\
        \Gamma, y : \tau_2 \proves E_2 : \tau
    }{ \Gamma \proves (\case(E)~\inl~x \mapsto E_1; \inr~y \mapsto E_2) : \tau }
\end{mathpar}
\end{frame}

\begin{frame}{Recap: Type Safety, Progress and Preservation}
    Below, let $E$ be a well-typed closed expression.

    \begin{theorem}[Type Safety]
        For all $E'$, if $E \to^* E'$, then either $E'$ is a value, or there is some $E''$ such that $E' \to E''$.
    \end{theorem}

    \begin{theorem}[Progress]
        Either $E$ is a value, or there is some $E'$ such that $E \to E'$.
    \end{theorem}

    \begin{theorem}[Preservation]
        If $E \to E'$, then $E'$ is well-typed (i.e., $\Gamma \proves E' : \tau$).
    \end{theorem}
\end{frame}

\begin{frame}{Proving Preservation}
    \begin{theorem}[Preservation]
        Let $E$ be a well-typed expression.
        If $E \to E'$, then $E'$ is well-typed.
    \end{theorem}
    \begin{proof}
        By induction on the derivation of $E \to E'$. \footnote{I will actually discuss the proof.}
    \end{proof}
\end{frame}

\begin{frame}{Proving Preservation: \textsc{Congr}}
    \textbf{Case \textsc{Congr}}: In this case, the proof of $E \to E'$ looks like:
    \begin{mathpar}
        \inferrule*[right=Congr]{
            \inferrule*[right=??]{
                \vdots
            }{ E_1 \to E_1' }
        }{ E_1~E_2 \to E_1'~E_2 }
    \end{mathpar}

    So $E = E_1~E_2$, $E' = E_1'~E_2$, and $E_1 \to E_1'$.
    By inversion, $\Gamma \proves E_1 : \tau_1 \to \tau_2$ and $\Gamma \proves E_2 : \tau_1$.
    Then by the induction hypothesis, $\Gamma \proves E_1' : \tau_1 \to \tau_2$, so by using the \textsc{App} rule, we have $\Gamma \proves E_1'~E_2 : \tau_2$.
\end{frame}

\begin{frame}{Proving Preservation: \textsc{$\beta$-reduce}}
    \textbf{Case \textsc{$\beta$-reduce}}: In this case, the proof of $E \to E'$ looks like:
    \begin{mathpar}
        \inferrule*[right=$\beta$-reduce]{
        }{ (\lambda x : \tau_1. E_1)~E_2 \to E_1[E_2/x] }
    \end{mathpar}

    So in this case, $E = (\lambda x : \tau_1. E_1)~E_2$ and $E' = E_1[E_2/x]$.

    By assumption, we have $\Gamma \proves (\lambda x : \tau_1. E_1) : \tau_1 \to \tau_2$ and $\Gamma \proves E_2 : \tau_1$.

    Then, by inversion of the $\textsc{Abs}$ rule, we also have $\Gamma, x : \tau_1 \proves E_1 : \tau_2$.

    But if $\Gamma, x : \tau_1 \proves E_1 : \tau_2$ and $\Gamma \proves E_2 : \tau_1$, then $\Gamma \proves E_1[E_2/x] : \tau_2$. \footnote{Technically, this must be proved too, but hopefully it's intuitive.}
\end{frame}

\begin{frame}{Empty Slide for Scratch Work}
\end{frame}

\begin{frame}{Proving Preservation: \textsc{$\alpha$-conv}}
    \textbf{Case \textsc{$\alpha$-conv}}: Left as an exercise.
\end{frame}

\begin{frame}{Proving Type Safety}
    \begin{itemize}
        \item Why did we do that?
    \end{itemize}
\end{frame}

\begin{frame}{Proving Type Safety}
    \begin{itemize}
        \item Why did we do that?
        \item So you understand why I keep mentioning reducing the number of cases to check (even worse if we did it with all the product/sum type rules!)
    \end{itemize}
\end{frame}

\begin{frame}{Proving Type Safety}
    \begin{itemize}
        \item Why did we do that?
        \item \st{So you understand why I keep mentioning reducing the number of cases to check}
        \item To show that our language definition is ``sensible'':
            \begin{itemize}
                \item We didn't forget any evaluation rules
                \item Typechecking guarantees that our programs run successfully
            \end{itemize}
    \end{itemize}
\end{frame}

\begin{frame}[fragile]{System $F$}
    \begin{itemize}
        \item Almost Haskell's type system (\textbf{almost} System $F_\omega$ \footnote{As the name suggests, it builds heavily on System $F$})
        \item Basically, it's simply-typed $\lambda$-calculus, but with \emph{parametric polymorphism}
        \item Below is the identity function in Haskell\footnote{You technically need \lstinline[language=haskell, basicstyle=\small\ttfamily]{-XExplicitForAll} to write it this way.}
\begin{lstlisting}[language=haskell, basicstyle=\small\ttfamily]
id : forall a. a -> a
id x = x
\end{lstlisting}
        \item And in System $F$:
        \begin{align*}
            \text{id} &: \forall \alpha. \alpha \to \alpha \\
            \text{id} &= (\Lambda \alpha. \lambda x : \alpha. x)
        \end{align*}
    \end{itemize}
\end{frame}

\begin{frame}[fragile]{System $F$ Syntax}
    \begin{itemize}
        \item System $F$ is again based on the $\lambda$-calculus, but with a much more powerful type system
            \begin{itemize}
                \item We've dropped the built-in types altogether, because System $F$ is powerful enough to define them (including products and sums)
                \item It's deceptively simple---the rules will probably be confusing because it's not clear exactly how they work or why they exist.
            \end{itemize}
        \item We use $\alpha$, $\beta$, $\gamma$, etc. for \emph{type variables}
    \end{itemize}
\begin{tabular}{l r l l}
    $\tau$ & \bnfdef & $\alpha$ \bnfalt $\tau \to \tau$ \bnfalt $\forall \alpha. \tau$ \\
    E & \bnfdef & $x$ \bnfalt $\lambda x : \tau. E$ \bnfalt $E~E$ \bnfalt $\Lambda \alpha. E$ \bnfalt $E[\tau]$
\end{tabular}
\end{frame}

\begin{frame}[fragile]{System $F$ Evaluation}
    \begin{itemize}
        \item The evaluation rules are the same as before, except we've added one new rule to deal with the new expressions
        \item We overload the substitution syntax to allow substituting for type variables
        \item Technically, we only need to do the substitution so the proof works---we can otherwise erase types at runtime if we cared about efficiency
    \end{itemize}

    \begin{mathpar}
        \inferrule*[right=App-Ty]{
        }{ (\Lambda \alpha. E)[\tau] \to E[\tau/\alpha] }
    \end{mathpar}
\end{frame}

\begin{frame}[fragile]{System $F$ Typechecking}
    \begin{itemize}
        \item We only need two new type rules
        \item Extend type contexts to contain type variables to ensure we don't bind the same type variable twice
    \end{itemize}

    \begin{mathpar}
        \inferrule*[right=T-Abs]{
            \Gamma, \alpha \proves E : \tau
        }{ \Gamma \proves (\Lambda \alpha. E) : \forall \alpha. \tau }

        \inferrule*[right=T-App]{
            \Gamma \proves E : \forall \alpha. \sigma
        }{ \Gamma \proves E[\tau] : \sigma[\tau / \alpha] }
    \end{mathpar}
\end{frame}

\begin{frame}[fragile]{System $F$ Typechecking: Type Formation}
    \begin{mathpar}
        \inferrule*[right=T-Ok-Var]{
        }{ \Gamma, \alpha \proves \alpha }

        \inferrule*[right=T-Ok-Abs]{
            \Gamma \proves \tau_1
            \\
            \Gamma \proves \tau_2
        }{ \Gamma \proves \tau_1 \to \tau_2 }

        \inferrule*[right=T-Ok-Forall]{
            \Gamma, \alpha \proves \tau
        }{ \Gamma \proves \forall \alpha. \tau }
    \end{mathpar}
\end{frame}

\begin{frame}[fragile]{Comments on System $F$}
    \begin{itemize}
        \item System $F$ actually is in some sense more powerful (well, differently powerful) than default Haskell because we can put $\forall$ anywhere
            \begin{itemize}
                \item Without extensions, \lstinline[language=haskell, basicstyle=\small\ttfamily]{forall} is only allowed at the ``top'' of types, because it makes type inference possible.
            \end{itemize}
        \item Type inference for System $F$ is undecidable\footnote{i.e., not having any type annotations; this is why Haskell doesn't let you put $\forall$ anywhere}.
    \end{itemize}
\end{frame}

\begin{frame}[fragile]{Comments on System $F$: Free Theorems}
    \begin{itemize}
        \item You may have heard of free theorems\footnote{If you read the paper you'll see it's not exactly ``free''. See: Philip Wadler. "Theorems for free!" FPCA (1989).}
        \item This refers to the idea that polymorphic types are so strong that just knowing the type of a (closed) term can tell you its bevahior
        \item For example:
            \begin{itemize}
                \item If $f : \forall \alpha. \alpha \to \alpha$, then $f \equiv \Lambda \alpha. \lambda x : \alpha. x$.
                \item If $f : \forall \alpha. \forall \beta. \alpha \times \beta \to \alpha$, $g : A \to B$, and $h : C \to D$ then $g \after f[A][C] = f[A][C] \after (g \times h)$.
\begin{lstlisting}[language=haskell, basicstyle=\small\ttfamily]
fst : (a,c) -> a
g : a -> b
h : c -> d
(x, y) : (a, c)
-- Then
g (fst (x, y)) = fst (g x, h y)
\end{lstlisting}
            \end{itemize}
    \end{itemize}
\end{frame}

\begin{frame}[fragile]{Maude Implementation}
    \begin{itemize}
        \item We can easily implement System $F$ in Maude
        \item First, the syntax; we declare sorts and subsorts (\texttt{Qid} is built-in and means ``Quoted Identifier'')
    \end{itemize}

\begin{lstlisting}[language=haskell, basicstyle=\small\ttfamily]
sorts Expr TyVar Type .
subsort Qid < Expr .
subsort TyVar < Type .

ops alpha beta gamma : -> TyVar .
\end{lstlisting}
\end{frame}

\begin{frame}[fragile]{Maude Implementation: Types and Expressions}
    \begin{itemize}
        \item Next, we give the syntax for types and expressions
    \end{itemize}

\begin{lstlisting}[language=haskell, basicstyle=\small\ttfamily]
op _->_ : Type Type -> Type .
op forall_._ : TyVar Type -> Type .

op lambda_:_._ : Qid Type Expr -> Expr .
op __ : Expr Expr -> Expr .
op /\_._ : TyVar Expr -> Expr .
op _[_] : Expr Type -> Expr .
\end{lstlisting}
\end{frame}

\begin{frame}[fragile]{Maude Implementation: Examples}
    \begin{itemize}
        \item Maude can now parse basic programs
        \item We'll define \texttt{bool} and \texttt{TRUE} shortly
    \end{itemize}

\begin{lstlisting}[language=haskell, basicstyle=\small\ttfamily]
(/\ alpha . lambda 'x : alpha . 'x) [ bool ] TRUE
\end{lstlisting}
\end{frame}

\begin{frame}[fragile]{Maude Implementation: Evaluation}
    \begin{itemize}
        \item Adding the evaluation rules is quite simple (if we ignore substitution)
        \item We don't need \textsc{Congr} because Maude does that for us
        \item I moved \textsc{$\alpha$-conv} into the substitution operation (not shown, because it's long and boring)
    \end{itemize}

\begin{lstlisting}[language=haskell, basicstyle=\small\ttfamily]
rl [beta-red]: (lambda X : T . E1) E2 => E1[E2 / X] .
rl [ty-app]: (/\ A . E) [ T ] => E[T / A] .
\end{lstlisting}
\end{frame}

\begin{frame}[fragile]{Maude Implementation: Typechecking}
    \begin{itemize}
        \item For typechecking, we add type environments and the typing judgement
    \end{itemize}

\begin{lstlisting}[language=haskell, basicstyle=\small\ttfamily]
sort TyEnv .
op nil : -> TyEnv .
op _:_ : Qid Type -> TyEnv .
op _,_ : TyEnv TyEnv -> TyEnv [comm assoc id: nil] .

op _|-_ : TyEnv Expr -> Type .
\end{lstlisting}
\end{frame}

\begin{frame}[fragile]{Maude Implementation: Typechecking Rules}
    \begin{itemize}
        \item Next, we add the rules from the simple-typed $\lambda$-calculus
    \end{itemize}

\begin{lstlisting}[language=haskell, basicstyle=\small\ttfamily]
rl [var]: (Gamma, (X : T)) |- X => T .

crl [abs]: Gamma |- (lambda X : T1 . E2) => T1 -> T2
    if (Gamma, (X : T1)) |- E2 => T2 .

crl [app]: Gamma |- E1 E2 => T2
    if Gamma |- E2 => T1 /\
       Gamma |- E1 => T1 -> T2 .
\end{lstlisting}
\end{frame}

\begin{frame}[fragile]{Maude Implementation: Typechecking Rules}
    \begin{itemize}
        \item And finally, the System $F$ rules
    \end{itemize}

\begin{lstlisting}[language=haskell, basicstyle=\small\ttfamily]
crl [ty-app]: Gamma |- E[T2] => T1[T2 / A]
    if Gamma |- E => forall A . T1 .

crl [ty-abs]: Gamma |- /\ A . E => forall A . T
    if Gamma |- E => T .
\end{lstlisting}
\end{frame}

\begin{frame}[fragile]{Maude Implementation: Typechecking Example}
    \begin{itemize}
        \item We can now use this to typecheck programs:
    \end{itemize}

\footnotesize
\begin{lstlisting}[language=haskell, basicstyle=\small\ttfamily]
rewrite nil |- /\ alpha . lambda 'x : alpha . 'x .
rewrites: 3 in 0ms cpu (0ms real) (~ rewrites/second)
result Type: forall alpha . (alpha -> alpha)
==========================================
rewrite nil |- /\ alpha . /\ beta .
               lambda 'x : alpha .
               lambda 'y : beta . 'x .
rewrites: 5 in 0ms cpu (0ms real) (~ rewrites/second)
result Type: forall alpha .
             forall beta . (alpha -> beta -> alpha)
\end{lstlisting}
\end{frame}

\begin{frame}[fragile]{Church Encodings}
    \begin{itemize}
        \item We talked briefly about these before
\begin{align*}
    \texttt{True} & := (\lambda x. \lambda y. x) \\
    \texttt{False} & := (\lambda x. \lambda y. y) \\
    \texttt{And} & := (\lambda b_1. \lambda b_2. b_1~b_2~\texttt{False})
\end{align*}
        \item The basic idea about encoding some type $T$ is to think about how you use values of type $T$, then write a type that reflects that
    \end{itemize}
\end{frame}

\begin{frame}[fragile]{Church Encodings: (Recap) Booleans}
    \begin{itemize}
        \item Encoding \texttt{bool}: how do you use booleans? Mostly as conditions
        \item Abstracting away various kinds of conditions (if statements, while loops, for loops, etc.), we just need if statements and recursion
        \item So the way to use a boolean value \texttt{b} is basically:
\begin{lstlisting}[language=haskell, basicstyle=\small\ttfamily]
if b then val1 else val2
\end{lstlisting}
        \item So if \texttt{b} is true, then we return \texttt{val1}, otherwise we return \texttt{val2}
    \end{itemize}
\begin{align*}
    \texttt{True} & := (\lambda x. \lambda y. x) \\
    \texttt{False} & := (\lambda x. \lambda y. y)
\end{align*}
\end{frame}

\begin{frame}[fragile]{Church Encodings: Booleans in System $F$}
    \begin{itemize}
        \item Previously, we couldn't type these functions (because they take arguments of any types), now we can
        \item Any guesses on how to write in System $F$?
    \end{itemize}
\begin{align*}
    \texttt{True} & := (\lambda x. \lambda y. x) \\
    \texttt{False} & := (\lambda x. \lambda y. y) \\
    \texttt{And} & := (\lambda b_1. \lambda b_2. b_1~b_2~\texttt{False})
\end{align*}
\end{frame}

\begin{frame}[fragile]{Church Encodings: Booleans in System $F$}
    \begin{itemize}
        \item Previously, we couldn't type these functions (because they take arguments of any types), now we can
        \item We use the same type for both arguments so we know the return type statically
    \end{itemize}
\begin{align*}
    \texttt{True} & : \forall \alpha. \alpha \to \alpha \to \alpha \\
    \texttt{True} & := \Lambda \alpha. \lambda x : \alpha. \lambda y : \alpha. x \\
    \texttt{False} & : \forall \alpha. \alpha \to \alpha \to \alpha \\
    \texttt{False} & := \Lambda \alpha. \lambda x : \alpha. \lambda y : \alpha. y
\end{align*}
\end{frame}

\begin{frame}[fragile]{Maude Implementation: Church Encodings (Bool)}

\footnotesize
\begin{lstlisting}[language=haskell, basicstyle=\small\ttfamily]
op bool : -> Type .
eq bool = forall alpha . alpha -> alpha -> alpha .

ops TRUE FALSE : -> Expr .
eq TRUE = /\ alpha .
    lambda 'x : alpha . lambda 'y : alpha . 'x .
eq FALSE = /\ alpha .
    lambda 'x : alpha . lambda 'y : alpha . 'y .

op AND : -> Expr .
eq AND = lambda 'b1 : bool .
         lambda 'b2 : bool .
            ((('b1 [ bool ]) 'b2) FALSE) .
\end{lstlisting}
\end{frame}

\begin{frame}[fragile]{Maude Implementation: Church Encodings (Bool) Example}
\footnotesize
\begin{lstlisting}[language=haskell, basicstyle=\small\ttfamily]
rewrite (AND TRUE) ((AND FALSE) TRUE) .
rewrites: 674 in 0ms cpu (0ms real) (~ rewrites/second)
result Expr: /\ alpha .
    lambda '0 : alpha . lambda '1 : alpha . '1
\end{lstlisting}
\end{frame}

\begin{frame}[fragile]{Church Encodings: Pairs}
    \begin{itemize}
        \item Before, we had to laboriously add new syntax and rules to add pairs
        \item In System $F$, we don't need any of this!
        \item How do you use a pair? Well, you use the two values inside the pair.
        \item In Haskell:
    \end{itemize}
\begin{lstlisting}[language=haskell, basicstyle=\small\ttfamily]
mkPair :: a -> b -> (a -> b -> c) -> c
mkPair a b = \f -> f a b
\end{lstlisting}
\end{frame}

\begin{frame}[fragile]{Church Encodings: Pairs}
    \begin{itemize}
        \item In System $F$ (Do not be afraid):
\begin{align*}
    \texttt{mkPair} &: \forall \alpha. \forall \beta. \alpha \to \beta \to \forall \gamma. (\alpha \to \beta \to \gamma) \to \gamma \\
    \texttt{mkPair} &= \Lambda \alpha. \Lambda \beta. \lambda a : \alpha. \lambda b : \beta. \Lambda \gamma. \lambda f : \alpha \to \beta \to \gamma. f~a~b
\end{align*}
        \item So
\[
    \alpha \times \beta := \forall \gamma. (\alpha \to \beta \to \gamma) \to \gamma
\]
        \item Using this abbreviation (which we can't formalize without higher-kinded types)
\[
    \texttt{mkPair} : \forall \alpha. \forall \beta. \alpha \to \beta \to \alpha \times \beta
\]
    \end{itemize}
\begin{lstlisting}[language=haskell, basicstyle=\small\ttfamily]
mkPair :: a -> b -> (a -> b -> c) -> c
mkPair a b = \f -> f a b
\end{lstlisting}
\end{frame}

\begin{frame}[fragile]{Church Encodings: Pairs (Examples)}
    \begin{itemize}
        \item Programs using pairs:
\begin{align*}
    \texttt{fst} & : \forall \alpha. \forall \beta. \alpha \times \beta \to \alpha \\
    \texttt{fst} & = ???
\end{align*}
    \end{itemize}
\end{frame}

\begin{frame}[fragile]{Church Encodings: Pairs (Examples)}
    \begin{itemize}
        \item Some example programs; we'll do more later, but we can't write a ton of programs with \textbf{only} pairs.
\begin{align*}
    \texttt{fst} & : \forall \alpha. \forall \beta. \alpha \times \beta \to \alpha \\
    \texttt{fst} & = \Lambda \alpha, \beta. \forall p : \alpha \times \beta. p[\alpha]~(\lambda x : \alpha, y : \beta. x) \\
    \texttt{snd} & : \forall \alpha. \forall \beta. \alpha \times \beta \to \beta \\
    \texttt{snd} & = \Lambda \alpha, \beta. \forall p : \alpha \times \beta. p[\beta]~(\lambda x : \alpha, y : \beta. y)
\end{align*}
    \end{itemize}
\end{frame}

\begin{frame}{Church Encodings: Evaluating \texttt{fst}}
\footnotesize
\begin{align*}
    &\texttt{fst}[\N][\bool]~(\texttt{mkPair}[\N][\bool]~4~\false) \\
    & \to (\Lambda \alpha, \beta. \forall p : \alpha \times \beta. p[\alpha]~(\lambda x : \alpha, y : \beta. x))[\N][\bool]~(\texttt{mkPair}[\N][\bool]~4~\false) \\
    & \to (\Lambda \beta. \forall p : \N \times \beta. p[\N]~(\lambda x : \N, y : \beta. x))[\bool]~(\texttt{mkPair}[\N][\bool]~4~\false) \\
    & \to (\forall p : \N \times \bool. p[\N]~(\lambda x : \N, y : \bool. x))~(\texttt{mkPair}[\N][\bool]~4~\false) \\
    & \to (\texttt{mkPair}[\N][\bool]~4~\false)[\N]~(\lambda x : \N, y : \bool. x) \\
    & \to ((\Lambda \alpha, \beta. \lambda a : \alpha, b : \beta. \Lambda \gamma. \lambda f : \alpha \to \beta \to \gamma. f~a~b)[\N][\bool]~4~\false)[\N]~(\lambda x : \N, y : \bool. x) \\
    & \to ((\lambda a : \N, b : \bool. \Lambda \gamma. \lambda f : \N \to \bool \to \gamma. f~a~b)~4~\false)[\N]~(\lambda x : \N, y : \bool. x) \\
    & \to (\lambda f : \N \to \bool \to \N. f~4~\false)~(\lambda x : \N, y : \bool. x) \\
    & \to (\lambda x : \N, y : \bool. x)~4~\false \\
    & \to (\lambda y : \bool. 4)~\false \\
    & \to 4 \\
\end{align*}
\end{frame}

\begin{frame}[fragile]{Church Encodings: Lists}
    \begin{itemize}
        \item Before, we had to laboriously add new syntax and rules to add pairs
        \item How do you use a list? Well, you iterate through the list and use every element (often called a \emph{fold})
    \end{itemize}

\begin{lstlisting}[language=haskell, basicstyle=\small\ttfamily]
foldr f init [] = init
foldr f init (x:xs) = f x (foldr f init xs)
\end{lstlisting}

    \small
    \begin{align*}
        \texttt{List}~\alpha & = \forall \beta. (\alpha \to \beta \to \beta) \to \beta \to \beta \\
        \texttt{nil} & : \forall \alpha. \texttt{List}~\alpha \\
        \texttt{nil} & = \Lambda \alpha. \Lambda \beta. \lambda f : \alpha \to \beta \to \beta. \lambda x : \beta . x \\
        \texttt{cons} & : \forall \alpha. \alpha \to \texttt{List}~\alpha \to \texttt{List}~\alpha \\
        \texttt{cons} & = \Lambda \alpha. \lambda x : \alpha. \lambda xs : \texttt{List}~\alpha . \Lambda \beta. \lambda f : \alpha \to \beta \to \beta . \lambda y : \beta . f~x~(xs[\beta]~f~y) \\
        \texttt{sum} & : \texttt{List}~\N \to \N \\
        \texttt{sum} & = \lambda xs : \texttt{List}~\N . xs[\N]~(\lambda x, y : \N. x + y)~0
    \end{align*}
\end{frame}

\begin{frame}[fragile]{Church Encodings: Lists Example}
    \begin{align*}
        & \texttt{sum}~(\texttt{cons}~[\N]~1~(\texttt{cons}~[\N]~2~(\texttt{nil}~[\N]))) \\
        & (\texttt{cons}~[\N]~1~(\texttt{cons}~[\N]~2~(\texttt{nil}~[\N])))[\N]~(\lambda x, y : \N. x + y)~0 \\
        & \to (\lambda x : \N. \lambda xs : \texttt{List}~\N . \Lambda \beta. \lambda f : \N \to \beta \to \beta . \lambda y : \beta . f~x~(xs[\beta]~f~y))~1~(\texttt{cons}~[\N]~2~(\texttt{nil}~[\N]))[\N]~(\lambda x, y : \N. x + y)~0 \\
        & \to (\lambda xs : \texttt{List}~\N . \Lambda \beta. \lambda f : \N \to \beta \to \beta . \lambda y : \beta . f~1~(xs[\beta]~f~y))~(\texttt{cons}~[\N]~2~(\texttt{nil}~[\N]))[\N]~(\lambda x, y : \N. x + y)~0 \\
        & \to (\Lambda \beta. \lambda f : \N \to \beta \to \beta . \lambda y : \beta . f~1~((\texttt{cons}~[\N]~2~(\texttt{nil}~[\N]))[\beta]~f~y))[\N]~(\lambda x, y : \N. x + y)~0 \\
        & \to (\lambda y : \N . 1 + ((\texttt{cons}~[\N]~2~(\texttt{nil}~[\N]))[\N]~f~y))~0 \\
        & \to 1 + ((\texttt{cons}~[\N]~2~(\texttt{nil}~[\N]))[\N]~f~0) \\
        & \vdots \\
        & \to 1 + 2 + 0
    \end{align*}
\end{frame}

\begin{frame}{Featherweight Java}
    \begin{itemize}
        \item Featherweight Java (FJ) is basically the smallest interesting subset of Java
        \item Important features: classes, methods, fields, \textbf{inheritance}
        \item But \textbf{no}: mutation, generics, built-ins (other than \texttt{Object}), access modifiers, interfaces, etc.
    \end{itemize}

\begin{tabular}{l r l l}
    \texttt{L} & \bnfdef & \texttt{class $C$ extends $C$ \{ $\overline{C~f}$; $\overline{M}$ \} } & (class def.) \\
    \texttt{M} & \bnfdef & \texttt{$C$ m($\overline{C~x}$) \{ return $e$; \} } & (method def.) \\
    \texttt{e} & \bnfdef & \texttt{$x$} \bnfalt \texttt{$e.f$} \bnfalt \texttt{$e.m(\overline{e})$} \bnfalt \texttt{new $C$($\overline{e}$)} \bnfalt \texttt{($C$)$e$} & (expressions) \\
    \texttt{P} & \bnfdef & \texttt{$\overline{\texttt{L}}$; $e$} & (programs)
\end{tabular}
\end{frame}

\begin{frame}[fragile]{A Program in FJ}
\begin{lstlisting}[language=java, basicstyle=\scriptsize\ttfamily]
class If extends Object {
  Object if(Object a, Object b) { return a; }
}
class IfFalse extends If {
  Object if(Object a, Object b) { return b; }
}
class Integer extends Object {
  // ...
  Integer sub(Integer other) { return /* ... */; }
  Integer mul(Integer other) { return /* ... */; }
  If ifZero() { return /* ... */; }
}
class Main extends Object {
  Integer fact(Integer x) {
// Technically, must delay computation of the 2nd argument
    return (Integer)x.isZero().if(1,x.mul(this.fact(x.sub(1))));
  }
};
new Main().factorial(6) // == 720
\end{lstlisting}
\end{frame}

\begin{frame}[fragile]{FJ: Big Ideas}
    \begin{itemize}
        \item Declarations: Programs in $\lambda$-calculus and System $F$ are expressions---FJ is more like real programming languages
        \item Inheritance and Casting: We recover the ability to write any program with casts, but still have some type-safety because of inheritance
        \item Type Soundness: More complicated, like in real languages
        \item Things to note:
            \begin{itemize}
                \item In some sense, FJ is a ``re-skin'' of the $\lambda$-calculus
                \item Everything is a sub-class of \texttt{Object} (like in real Java), which gets ``special treatment''
                \item No statements: method bodies are expressions
                \item We'll talk about lnaguages with mutable environments next, and you'll wish we hadn't
            \end{itemize}
    \end{itemize}
\end{frame}

\begin{frame}[fragile]{FJ: Subtyping}
    \begin{itemize}
        \item We write $C <: D$ to mean ``$C$ is a subtype of $D$'' (in FJ, this is basically ``inherits from'')
        \item Note that we treat declarations as judgements\footnote{We could have a separate environment, but that's only really useful when the environment can change.}
    \end{itemize}

\begin{mathpar}
    \inferrule*[right=Sub-Refl]{
    }{ C <: C }

    \inferrule*[right=Sub-Trans]{
        C <: D
        \\
        D <: E
    }{ C <: E }

    \inferrule*[right=Sub-Extends]{
        \class~C~\extends~D~\{ \ldots \}
    }{ C <: D }
\end{mathpar}
\end{frame}

\begin{frame}[fragile]{FJ: Evaluation}
    \begin{itemize}
        \item In FJ, values are of the form $v \bnfdef \newc~C(\overline{v})$
            \begin{itemize}
                \item Constructors are essentially tuples
                \item They don't get evaluated
            \end{itemize}
    \end{itemize}

\begin{mathpar}
\inferrule*[right=E-Field]{
    \fields(C) = \overline{D~f}
}{ (\newc~C(\overline{v})).f_i \to v_i }

\inferrule*[right=E-Cast]{
    C <: D
}{ (D)(\newc~C(\overline{v})) \to \newc~C(\overline{v}) }

\inferrule*[right=E-Method]{
    \method(C, m) = D~m(\overline{A~x})~\{ \return~e; \}
}{ (\newc~C(\overline{v})).m(\overline{w}) \to e[\overline{w}/\overline{x}, \newc~C(\overline{v})/\this]}
\end{mathpar}
\end{frame}

\begin{frame}[fragile]{FJ: Typechecking}
    \begin{itemize}
        \item Typechecking is pretty similar to previous systems we've seen
        \item We still write $\Gamma \proves e : C$ to say that $e$ has type $C$ (some class) in the environment $\Gamma$
        \item Variables are exactly the same as before; fields require looking up in the ambient environment
    \end{itemize}

\begin{mathpar}
    \inferrule*[right=T-Var]{
    }{ \Gamma, x : C \proves x : C }

    \inferrule*[right=T-Field]{
        \Gamma \proves e : C
        \\
        \fields(C) = \overline{D~f}
    }{ \Gamma \proves e.f_i : D_i }
\end{mathpar}
\end{frame}

\begin{frame}[fragile]{FJ: Typechecking (Invoke, New)}
\begin{mathpar}
    \inferrule*[right=T-Invoke]{
        \Gamma \proves e : C
        \\
        \method(C, m) = D~m(\overline{A~x})~\{ \return~e_m; \}
        \\
        \Gamma \proves \overline{e' : A}
    }{ \Gamma \proves e.m(\overline{e'}) : D }

    \inferrule*[right=T-New]{
        \fields(C) = \overline{D~f}
        \\
        \Gamma \proves \overline{e : D}
    }{ \Gamma \proves \newc~C(\overline{e}) : C }
\end{mathpar}
\end{frame}

\begin{frame}[fragile]{FGJ}
    \begin{itemize}
        \item TODO
    \end{itemize}
\end{frame}

\end{document}

