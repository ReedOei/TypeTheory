\documentclass[leqno,presentation,usenames,dvipsnames]{beamer}
\DeclareGraphicsExtensions{.eps,.jpg,.png,.tif}
\usepackage{amssymb, amsmath, pdfpages, amsfonts, calc, times, type1cm, latexsym, xcolor, colortbl, hyperref, bookmark}
\usepackage{mathtools}
\usepackage{bm}
\usepackage{graphicx}
\usepackage{soul}
\usepackage{tabularx}
\usepackage{multirow}
\usepackage{listings}
\usepackage{mathpartir}
\usepackage{xspace}

\usepackage[latin1]{inputenc}
\usepackage[english]{babel}

\usetheme{Szeged}
\usecolortheme{beaver}

\definecolor{websiteGreen}{RGB}{107, 224, 134}
\definecolor{silvery}{RGB}{232, 241, 248}
\definecolor{deepOrange}{RGB}{209, 126, 0}

\definecolor{softTan}{RGB}{240, 240, 223}
\definecolor{deepGreen}{RGB}{87, 149, 115}
\definecolor{lilac}{RGB}{199, 164, 202}
\definecolor{lightOlive}{RGB}{168, 166, 96}
\definecolor{deepOlive}{RGB}{101, 109, 41}

\setbeamercolor*{palette primary}{bg=websiteGreen}
\setbeamercolor*{palette secondary}{bg=white}
\setbeamercolor*{palette tertiary}{bg=silvery}
\setbeamercolor*{palette quaternary}{bg=websiteGreen}

\setbeamercolor{section in head/foot}{fg=deepOrange}
\setbeamerfont{section in head/foot}{series=\bfseries}

\setbeamercolor{titlelike}{parent=palette primary,bg=websiteGreen,fg=white}
\setbeamercolor{frametitle}{parent=palette primary,bg=websiteGreen, fg=white}

\addtobeamertemplate{frametitle}{\vspace*{-0.65\baselineskip}}{}

\newcommand{\tryS}{\textbf{\texttt{try}}\xspace}
\newcommand{\Int}{\textbf{Int}\xspace}
\newcommand{\ok}{\textbf{ok}\xspace}
\newcommand{\catchS}{\textbf{\texttt{catch}}\xspace}

\newcommand{\inl}{\textbf{\texttt{inl}}\xspace}
\newcommand{\inr}{\textbf{\texttt{inr}}\xspace}
\newcommand{\case}{\textbf{\texttt{case}}\xspace}

\newcommand{\bool}{\textbf{Bool}}
\newcommand{\fin}{\textbf{Fin}}
\newcommand{\newc}{\texttt{new}}
\newcommand{\class}{\texttt{class}}
\newcommand{\extends}{\texttt{extends}}
\newcommand{\this}{\texttt{this}}
\newcommand{\fields}{\texttt{fields}}
\newcommand{\method}{\texttt{method}}
\newcommand{\return}{\texttt{return}}

\newtheorem*{conjecture}{Conjecture}
\newtheorem*{proposition}{Proposition}

\title{SIGPLAN}
\subtitle{Type Theory (part $4$ of $n$): Featherweight Java}
\date{}

\input{LaTeX/macros.tex}

\begin{document}

\frame{\titlepage}

\begin{frame}{Recap}
    \begin{itemize}
        \item Last time, we talked about System $F$ and Church Encodings
        \item TODO: ADD MORE INTERACTION?
    \end{itemize}
\end{frame}

\begin{frame}[fragile]{Recap: System $F$}
    \begin{itemize}
        \item Basically, simply-typed $\lambda$-calculus, but with \emph{parametric polymorphism}
        \item Identity function in System $F$:
        \begin{align*}
            \text{id} &: \forall \alpha. \alpha \to \alpha \\
            \text{id} &= (\Lambda \alpha. \lambda x : \alpha. x)
        \end{align*}
    \end{itemize}

    \begin{tabular}{l r l l}
        $\tau$ & \bnfdef & $\alpha$ \bnfalt $\tau \to \tau$ \bnfalt $\forall \alpha. \tau$ \\
        E & \bnfdef & $x$ \bnfalt $\lambda x : \tau. E$ \bnfalt $E~E$ \bnfalt $\Lambda \alpha. E$ \bnfalt $E[\tau]$
    \end{tabular}

    \begin{mathpar}
        \inferrule*[right=App-Ty]{
        }{ (\Lambda \alpha. E)[\tau] \to E[\tau/\alpha] }
    \end{mathpar}

    \begin{mathpar}
        \inferrule*[right=T-Abs]{
            \Gamma, \alpha \proves E : \tau
        }{ \Gamma \proves (\Lambda \alpha. E) : \forall \alpha. \tau }

        \inferrule*[right=T-App]{
            \Gamma \proves E : \forall \alpha. \sigma
        }{ \Gamma \proves E[\tau] : \sigma[\tau / \alpha] }
    \end{mathpar}
\end{frame}

\begin{frame}[fragile]{Recap: Church Encodings: Booleans}
    \begin{itemize}
        \item In the untyped $\lambda$-calculus
            \begin{align*}
                \texttt{True} & := (\lambda x. \lambda y. x) \\
                \texttt{False} & := (\lambda x. \lambda y. y) \\
                \texttt{And} & := (\lambda b_1. \lambda b_2. b_1~b_2~\texttt{False})
            \end{align*}
        \item In System $F$:
            \begin{align*}
                \texttt{True} & : \forall \alpha. \alpha \to \alpha \to \alpha \\
                \texttt{True} & := \Lambda \alpha. \lambda x : \alpha. \lambda y : \alpha. x \\
                \texttt{False} & : \forall \alpha. \alpha \to \alpha \to \alpha \\
                \texttt{False} & := \Lambda \alpha. \lambda x : \alpha. \lambda y : \alpha. y \\
                \texttt{And} & : \bool \to \bool \to \bool \\
                \texttt{And} & := \lambda x : \bool. \lambda y: \bool. \Lambda \alpha. x[\bool]~y~\texttt{False}
            \end{align*}
    \end{itemize}
\end{frame}

\begin{frame}[fragile]{Church Encodings: Pairs}
    \begin{itemize}
        \item Before, we had to laboriously add new syntax and rules to add pairs
        \item In System $F$, we don't need any of this!
        \item How do you use a pair? Well, you use the two values inside the pair.
        \item In Haskell:
    \end{itemize}
\begin{lstlisting}[language=haskell, basicstyle=\small\ttfamily]
mkPair :: a -> b -> (a -> b -> c) -> c
mkPair a b = \f -> f a b
\end{lstlisting}
\end{frame}

\begin{frame}[fragile]{Church Encodings: Pairs}
    \begin{itemize}
        \item In System $F$ (Do not be afraid):
\begin{align*}
    \texttt{mkPair} &: \forall \alpha. \forall \beta. \alpha \to \beta \to \forall \gamma. (\alpha \to \beta \to \gamma) \to \gamma \\
    \texttt{mkPair} &= \Lambda \alpha. \Lambda \beta. \lambda a : \alpha. \lambda b : \beta. \Lambda \gamma. \lambda f : \alpha \to \beta \to \gamma. f~a~b
\end{align*}
        \item So
\[
    \alpha \times \beta := \forall \gamma. (\alpha \to \beta \to \gamma) \to \gamma
\]
        \item Using this abbreviation (which we can't formalize without higher-kinded types)
\[
    \texttt{mkPair} : \forall \alpha. \forall \beta. \alpha \to \beta \to \alpha \times \beta
\]
    \end{itemize}
\begin{lstlisting}[language=haskell, basicstyle=\small\ttfamily]
mkPair :: a -> b -> (a -> b -> c) -> c
mkPair a b = \f -> f a b
\end{lstlisting}
\end{frame}

\begin{frame}[fragile]{Church Encodings: Pairs (Examples)}
    \begin{itemize}
        \item Programs using pairs:
\begin{align*}
    \texttt{fst} & : \forall \alpha. \forall \beta. \alpha \times \beta \to \alpha \\
    \texttt{fst} & = ???
\end{align*}
    \end{itemize}
\end{frame}

\begin{frame}[fragile]{Church Encodings: Pairs (Examples)}
    \begin{itemize}
        \item Some example programs; we'll do more later, but we can't write a ton of programs with \textbf{only} pairs.
\begin{align*}
    \texttt{fst} & : \forall \alpha. \forall \beta. \alpha \times \beta \to \alpha \\
    \texttt{fst} & = \Lambda \alpha, \beta. \forall p : \alpha \times \beta. p[\alpha]~(\lambda x : \alpha, y : \beta. x) \\
    \texttt{snd} & : \forall \alpha. \forall \beta. \alpha \times \beta \to \beta \\
    \texttt{snd} & = \Lambda \alpha, \beta. \forall p : \alpha \times \beta. p[\beta]~(\lambda x : \alpha, y : \beta. y)
\end{align*}
    \end{itemize}
\end{frame}

\begin{frame}{Church Encodings: Evaluating \texttt{fst}}
\footnotesize
\begin{align*}
    &\texttt{fst}[\N][\bool]~(\texttt{mkPair}[\N][\bool]~4~\false) \\
    & \to (\Lambda \alpha, \beta. \forall p : \alpha \times \beta. p[\alpha]~(\lambda x : \alpha, y : \beta. x))[\N][\bool]~(\texttt{mkPair}[\N][\bool]~4~\false) \\
    & \to (\Lambda \beta. \forall p : \N \times \beta. p[\N]~(\lambda x : \N, y : \beta. x))[\bool]~(\texttt{mkPair}[\N][\bool]~4~\false) \\
    & \to (\forall p : \N \times \bool. p[\N]~(\lambda x : \N, y : \bool. x))~(\texttt{mkPair}[\N][\bool]~4~\false) \\
    & \to (\texttt{mkPair}[\N][\bool]~4~\false)[\N]~(\lambda x : \N, y : \bool. x) \\
    & \to ((\Lambda \alpha, \beta. \lambda a : \alpha, b : \beta. \Lambda \gamma. \lambda f : \alpha \to \beta \to \gamma. f~a~b)[\N][\bool]~4~\false)[\N]~(\lambda x : \N, y : \bool. x) \\
    & \to ((\lambda a : \N, b : \bool. \Lambda \gamma. \lambda f : \N \to \bool \to \gamma. f~a~b)~4~\false)[\N]~(\lambda x : \N, y : \bool. x) \\
    & \to (\lambda f : \N \to \bool \to \N. f~4~\false)~(\lambda x : \N, y : \bool. x) \\
    & \to (\lambda x : \N, y : \bool. x)~4~\false \\
    & \to (\lambda y : \bool. 4)~\false \\
    & \to 4 \\
\end{align*}
\end{frame}

\begin{frame}[fragile]{Church Encodings: Lists}
    \begin{itemize}
        \item Before, we had to laboriously add new syntax and rules to add pairs
        \item How do you use a list? Well, you iterate through the list and use every element (often called a \emph{fold})
    \end{itemize}

\begin{lstlisting}[language=haskell, basicstyle=\small\ttfamily]
foldr f init [] = init
foldr f init (x:xs) = f x (foldr f init xs)
\end{lstlisting}

    \small
    \begin{align*}
        \texttt{List}~\alpha & = \forall \beta. (\alpha \to \beta \to \beta) \to \beta \to \beta \\
        \texttt{nil} & : \forall \alpha. \texttt{List}~\alpha \\
        \texttt{nil} & = \Lambda \alpha. \Lambda \beta. \lambda f : \alpha \to \beta \to \beta. \lambda x : \beta . x \\
        \texttt{cons} & : \forall \alpha. \alpha \to \texttt{List}~\alpha \to \texttt{List}~\alpha \\
        \texttt{cons} & = \Lambda \alpha. \lambda x : \alpha. \lambda xs : \texttt{List}~\alpha . \Lambda \beta. \lambda f : \alpha \to \beta \to \beta . \lambda y : \beta . f~x~(xs[\beta]~f~y) \\
        \texttt{sum} & : \texttt{List}~\N \to \N \\
        \texttt{sum} & = \lambda xs : \texttt{List}~\N . xs[\N]~(\lambda x, y : \N. x + y)~0
    \end{align*}
\end{frame}

\begin{frame}[fragile]{Church Encodings: Lists Example}
    \begin{align*}
        & \texttt{sum}~(\texttt{cons}~[\N]~1~(\texttt{cons}~[\N]~2~(\texttt{nil}~[\N]))) \\
        & (\texttt{cons}~[\N]~1~(\texttt{cons}~[\N]~2~(\texttt{nil}~[\N])))[\N]~(\lambda x, y : \N. x + y)~0 \\
        & \to (\lambda x : \N. \lambda xs : \texttt{List}~\N . \Lambda \beta. \lambda f : \N \to \beta \to \beta . \lambda y : \beta . f~x~(xs[\beta]~f~y))~1~(\texttt{cons}~[\N]~2~(\texttt{nil}~[\N]))[\N]~(\lambda x, y : \N. x + y)~0 \\
        & \to (\lambda xs : \texttt{List}~\N . \Lambda \beta. \lambda f : \N \to \beta \to \beta . \lambda y : \beta . f~1~(xs[\beta]~f~y))~(\texttt{cons}~[\N]~2~(\texttt{nil}~[\N]))[\N]~(\lambda x, y : \N. x + y)~0 \\
        & \to (\Lambda \beta. \lambda f : \N \to \beta \to \beta . \lambda y : \beta . f~1~((\texttt{cons}~[\N]~2~(\texttt{nil}~[\N]))[\beta]~f~y))[\N]~(\lambda x, y : \N. x + y)~0 \\
        & \to (\lambda y : \N . 1 + ((\texttt{cons}~[\N]~2~(\texttt{nil}~[\N]))[\N]~f~y))~0 \\
        & \to 1 + ((\texttt{cons}~[\N]~2~(\texttt{nil}~[\N]))[\N]~f~0) \\
        & \vdots \\
        & \to 1 + 2 + 0
    \end{align*}
\end{frame}

\begin{frame}{Featherweight Java}
    \begin{itemize}
        \item Featherweight Java (FJ) is basically the smallest interesting subset of Java
        \item Important features: classes, methods, fields, \textbf{inheritance}
        \item But \textbf{no}: mutation, generics, built-ins (other than \texttt{Object}), access modifiers, interfaces, etc.
    \end{itemize}

\begin{tabular}{l r l l}
    \texttt{L} & \bnfdef & \texttt{class $C$ extends $C$ \{ $\overline{C~f}$; $\overline{M}$ \} } & (class def.) \\
    \texttt{M} & \bnfdef & \texttt{$C$ m($\overline{C~x}$) \{ return $e$; \} } & (method def.) \\
    \texttt{e} & \bnfdef & \texttt{$x$} \bnfalt \texttt{$e.f$} \bnfalt \texttt{$e.m(\overline{e})$} \bnfalt \texttt{new $C$($\overline{e}$)} \bnfalt \texttt{($C$)$e$} & (expressions) \\
    \texttt{P} & \bnfdef & \texttt{$\overline{\texttt{L}}$; $e$} & (programs)
\end{tabular}
\end{frame}

\begin{frame}[fragile]{A Program in FJ}
\begin{lstlisting}[language=java, basicstyle=\scriptsize\ttfamily]
class If extends Object {
  Object if(Object a, Object b) { return a; }
}
class IfFalse extends If {
  Object if(Object a, Object b) { return b; }
}
class Integer extends Object {
  // ...
  Integer sub(Integer other) { return /* ... */; }
  Integer mul(Integer other) { return /* ... */; }
  If ifZero() { return /* ... */; }
}
class Main extends Object {
  Integer fact(Integer x) {
// Technically, must delay computation of the 2nd argument
    return (Integer)x.isZero().if(1,x.mul(this.fact(x.sub(1))));
  }
};
new Main().factorial(6) // == 720
\end{lstlisting}
\end{frame}

\begin{frame}[fragile]{FJ: Big Ideas}
    \begin{itemize}
        \item Declarations: Programs in $\lambda$-calculus and System $F$ are expressions---FJ is more like real programming languages
        \item Inheritance and Casting: We recover the ability to write any program with casts, but still have some type-safety because of inheritance
        \item Type Soundness: More complicated, like in real languages
        \item Things to note:
            \begin{itemize}
                \item In some sense, FJ is a ``re-skin'' of the $\lambda$-calculus
                \item Everything is a sub-class of \texttt{Object} (like in real Java), which gets ``special treatment''
                \item No statements: method bodies are expressions
                \item We'll talk about languages with mutable environments soon\footnote{probably...if people are interested}, and you'll wish we hadn't
            \end{itemize}
    \end{itemize}
\end{frame}

\begin{frame}[fragile]{FJ: Subtyping}
    \begin{itemize}
        \item We write $C <: D$ to mean ``$C$ is a subtype of $D$'' (in FJ, this is basically ``inherits from'')
        \item Note that we treat declarations as judgements\footnote{We could have a separate environment, but that's only really useful when the environment can change.}
    \end{itemize}

\begin{mathpar}
    \inferrule*[right=Sub-Refl]{
    }{ C <: C }

    \inferrule*[right=Sub-Trans]{
        C <: D
        \\
        D <: E
    }{ C <: E }

    \inferrule*[right=Sub-Extends]{
        \class~C~\extends~D~\{ \ldots \}
    }{ C <: D }
\end{mathpar}
\end{frame}

\begin{frame}[fragile]{FJ: Evaluation}
    \begin{itemize}
        \item In FJ, values are of the form $v \bnfdef \newc~C(\overline{v})$
            \begin{itemize}
                \item Constructors are essentially tuples
                \item They don't get evaluated
            \end{itemize}
    \end{itemize}

\begin{mathpar}
\inferrule*[right=E-Field]{
    \fields(C) = \overline{D~f}
}{ (\newc~C(\overline{v})).f_i \to v_i }

\inferrule*[right=E-Cast]{
    C <: D
}{ (D)(\newc~C(\overline{v})) \to \newc~C(\overline{v}) }

\inferrule*[right=E-Method]{
    \method(C, m) = D~m(\overline{A~x})~\{ \return~e; \}
}{ (\newc~C(\overline{v})).m(\overline{w}) \to e[\overline{w}/\overline{x}, \newc~C(\overline{v})/\this]}
\end{mathpar}
\end{frame}

\begin{frame}[fragile]{FJ: (Simple) Example Program}
    \begin{itemize}
        \item Below is a very simple program that we'll go through as an example of the evaluation process (and also typechecking later)
        \item Note the casts on the last line
    \end{itemize}

\begin{lstlisting}[language=java, basicstyle=\small\ttfamily]
class Pair extends Object {
  Object left, right;
  Object getLeft() { return this.left; }
  Object getRight() { return this.right; }
  Pair setLeft(Object newLeft) {
    return new Pair(newLeft, this.right);
  }
  Pair setRight(Object newRight) {
    return new Pair(this.left, newRight)
  }
}
new Pair((Object)0, (Object)1).setLeft((Object)4).getLeft()
\end{lstlisting}
\end{frame}

\begin{frame}[fragile]{FJ: (Simple) Example Program}
    \begin{itemize}
        \item Below is a very simple program that we'll go through as an example of the evaluation process (and also typechecking later)
    \end{itemize}

\begin{lstlisting}[language=java, basicstyle=\small\ttfamily]
class Pair extends Object {
  Object left, right;
  Object getLeft() { return this.left; }
  Object getRight() { return this.right; }
  Pair setLeft(Object newLeft) {
    return new Pair(newLeft, this.right);
  }
  Pair setRight(Object newRight) {
    return new Pair(this.left, newRight)
  }
}
new Pair(0, 1).setLeft(4).getLeft()
\end{lstlisting}
\end{frame}

\begin{frame}[fragile]{FJ: Simple Example Program Evaluation}
    \begin{itemize}
        \item Abbreviate version of the program we actually need
    \end{itemize}
\begin{lstlisting}[language=java, basicstyle=\small\ttfamily]
class Pair extends Object {
  Object getLeft() { return this.left; }
  Pair setLeft(Object newLeft) {
    return new Pair(newLeft, this.right);
  }
}
new Pair(0, 1).setLeft(4).getLeft()
\end{lstlisting}
    \begin{align*}
        & \newc~\texttt{Pair}(0, 1).\texttt{setLeft}(4).\texttt{getLeft}()  \\
        & \to \newc~\texttt{Pair}(4, \newc~\texttt{Pair}(0,1).\texttt{right}).\texttt{getLeft}() \\
        & \to \newc~\texttt{Pair}(4, 1).\texttt{getLeft}() \\
        & \to \newc~\texttt{Pair}(4, 1).\texttt{left} \\
        & \to 4
    \end{align*}
\end{frame}

\begin{frame}[fragile]{FJ: Typechecking}
    \begin{itemize}
        \item Typechecking is pretty similar to previous systems we've seen
        \item We still write $\Gamma \proves e : C$ to say that $e$ has type $C$ (some class) in the environment $\Gamma$
        \item Variables are exactly the same as before; fields require looking up in the ambient environment
    \end{itemize}

\begin{mathpar}
    \inferrule*[right=T-Var]{
    }{ \Gamma, x : C \proves x : C }

    \inferrule*[right=T-Field]{
        \Gamma \proves e : C
        \\
        \fields(C) = \overline{D~f}
    }{ \Gamma \proves e.f_i : D_i }
\end{mathpar}
\end{frame}

\begin{frame}[fragile]{FJ: Typechecking (Invoke, New)}
    \begin{itemize}
        \item There rules are essentially the application and pair constructors for before, but for $n$-ary functions/$n$-tuples
    \end{itemize}

\begin{mathpar}
    \inferrule*[right=T-Invoke]{
        \Gamma \proves e : C
        \\
        \method(C, m) = D~m(\overline{A~x})~\{ \return~e_m; \}
        \\
        \Gamma \proves \overline{e' : A}
    }{ \Gamma \proves e.m(\overline{e'}) : D }

    \inferrule*[right=T-New]{
        \fields(C) = \overline{D~f}
        \\
        \Gamma \proves \overline{e : D}
    }{ \Gamma \proves \newc~C(\overline{e}) : C }
\end{mathpar}
\end{frame}

\begin{frame}[fragile]{FJ: Typechecking (Casting)}
    \begin{itemize}
        \item This is a critical part of FJ, because it allows ``generic'' programs (\`a la pre-generics Java)
        \item In addition to the standard up and down casts, there are also ``stupid casts'' (a technical necessity, due to how things evaluate)
    \end{itemize}

\begin{mathpar}
    \inferrule*[right=T-UCast]{
        \Gamma \proves e : D
        \\
        D <: C
    }{ \Gamma \proves (C)e : C }

    \inferrule*[right=T-DCast]{
        \Gamma \proves e : D
        \\
        C <: D
    }{ \Gamma \proves (C)e : C }

    \inferrule*[right=T-SCast]{
        \Gamma \proves e : D
        \\
        D \not<: C
        \\
        C \not<: D
        \\
        \text{stupid warning}
    }{ \Gamma \proves (C)e : C }
\end{mathpar}
\end{frame}

\begin{frame}[fragile]{FJ: Method Typing}
    \begin{itemize}
        \item We also need rules to check when a method is ``okay''
        \item This boils down to checking:
            \begin{itemize}
                \item It's return expression evaluates to the right type
                \item If the method appears in a superclass, it must override it properly (i.e., types match the overidden method)
            \end{itemize}
    \end{itemize}

\begin{mathpar}
    \inferrule*[right=T-Method]{
        \overline{A~x}, \this : C \proves e : D
        \\
        \class~C~\extends~C'~\{ \ldots \}
        \\
        m \in C' \implies \method(C', m) = D~m(\overline{A~x})~\{\ldots\}
    }{ D~m(\overline{A~x})~\{~\return~e;\}~\texttt{ok in}~C }
\end{mathpar}
\end{frame}

\begin{frame}[fragile]{FJ: Simple Example Program}
    \begin{itemize}
        \item Recall the simple program from before; now we will typecheck it (partially)
    \end{itemize}

\begin{lstlisting}[language=java, basicstyle=\small\ttfamily]
class Pair extends Object {
  Object left, right;
  Object getLeft() { return this.left; }
  Object getRight() { return this.right; }
  Pair setLeft(Object newLeft) {
    return new Pair(newLeft, this.right);
  }
  Pair setRight(Object newRight) {
    return new Pair(this.left, newRight)
  }
}
new Pair(0, 1).setLeft(4).getLeft()
\end{lstlisting}
\end{frame}

\begin{frame}[fragile]{FJ: Simple Example Typechecking: \texttt{setLeft}}
    \begin{itemize}
        \item First, we check \texttt{setLeft ok in Pair}
        \item Let $\Gamma = \texttt{Object}~x, \this : \texttt{Pair}$
    \end{itemize}

\Wider[4em]{
    \scriptsize
    \begin{mathpar}
        \inferrule*[right=T-Method]{
            \inferrule*[right=T-New]{
                \inferrule*[right=T-Var]{
                }{ \Gamma \proves x : \texttt{Object } }
                \\
                \inferrule*[right=T-Field]{
                    \inferrule*[right=T-Var]{
                    }{ \Gamma \proves \this : \texttt{Pair}  }
                }{ \Gamma \proves \texttt{this.r} : \texttt{Object} }
            }{ \Gamma \proves \newc~\texttt{Pair}(x, \texttt{this.r}) : \texttt{Pair} }
        }{ \texttt{Pair setL}(\texttt{Object}~x)~\{ \return~\newc~\texttt{Pair}(x, \texttt{this.r}); \}~\texttt{ok in Pair} }
    \end{mathpar}
}
\end{frame}

\begin{frame}[fragile]{FJ: Simple Example Typechecking: \texttt{setLeft}}
    \begin{itemize}
        \item Now, we check $\newc~\texttt{Pair}(0, 1).\texttt{setLeft}(4).\texttt{getLeft}() : \texttt{Object}$
        \item We're going to pretend we have some \textsc{Int} rule that lets us have integer constants
    \end{itemize}

\Wider[4em]{
    \scriptsize
    \begin{mathpar}
        \inferrule*[right=T-Invoke]{
            \inferrule*[right=T-Invoke]{
                \inferrule*[right=T-New]{
                    \inferrule*[right=Int]{
                    }{ \cdot \proves 0, 1 : \texttt{Object} }
                }{ \cdot \proves \newc~\texttt{Pair}(0, 1) : \texttt{Pair} }
                \\
                \inferrule*[right=Int]{
                }{ \cdot \proves 4 : \texttt{Object} }
            }{ \cdot \proves \newc~\texttt{Pair}(0, 1).\texttt{setLeft}(4) : \texttt{Pair} }
        }{ \cdot \proves \newc~\texttt{Pair}(0, 1).\texttt{setLeft}(4).\texttt{getLeft}() : \texttt{Object} }
    \end{mathpar}
}
\end{frame}

\begin{frame}[fragile]{FJ: Progress and Preservation}
    \begin{itemize}
        \item Progress and preservation are slightly more complicated to state in FJ than for System $F$ or the $\lambda$-calculus
    \end{itemize}

    \begin{theorem}[Subject Reduction\footnote{We would call this preservation}]
        If $\Gamma \proves e : C$ and $e \to e'$, then $\Gamma \proves e : C'$ for some $C' <: C$.
    \end{theorem}

    \begin{theorem}[Progress]
        Suppose $\Gamma \proves e : C$ for some closed $e$.
        Then either $e = \newc~C(\overline{v}$, or $e$ is stuck at a downcast (i.e., it contains $(D)\newc~C(\overline{v})$ where $C \not<: D$).
    \end{theorem}
\end{frame}

\begin{frame}[fragile]{Featherweight Generic Java}
    \begin{itemize}
        \item There's an extension of FJ to have generics (like Java does now), called FGJ
        \item Our example program becomes:
    \end{itemize}
\begin{lstlisting}[language=java, basicstyle=\small\ttfamily]
class Pair<L extends Object, R extends Object>
    extends Object {
  L left; R right;
  T getLeft() { return this.left; }
  R getRight() { return this.right; }
  Pair<L,R> setLeft(T newLeft) {
    return new Pair<L,R>(newLeft, this.right);
  }
  Pair<L,R> setRight(R newRight) {
    return new Pair<L,R>(this.left, newRight)
  }
}
new Pair<Int,Int>(0, 1).setLeft(4).getLeft() // : Int
\end{lstlisting}
\end{frame}

\begin{frame}[fragile]{Next Steps}
    \begin{itemize}
        \item FGJ is considerably more complicated to explain than FJ
        \item This leads me to ask about the topic for next meeting: after Dylan's presentation, so 12/6, I guess? Also will probably be the last meeting of the semester
        \item So for possible topics we have:
            \begin{itemize}
                \item FGJ
                \item Some other type theory topic:
                    \begin{itemize}
                        \item Recursive types
                        \item Higher-kinded types (``type constructors'' like Haskell)
                        \item Whatever else I think is cool
                    \end{itemize}
                \item Something not type theory: compilers, interpreters, program verification\footnote{not PL, per se, but I don't mind}, proof assistants
            \end{itemize}
        \item We'll continue meeting next semester, but potentially at a different time (or maybe not).
            Also, I'll be graduating soon, so if someone is interested in taking over let me know (pls)
    \end{itemize}
\end{frame}

\begin{frame}{Empty Slide for Scratch Work}
\end{frame}

\end{document}

