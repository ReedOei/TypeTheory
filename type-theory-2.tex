\documentclass[leqno,presentation,usenames,dvipsnames]{beamer}
\DeclareGraphicsExtensions{.eps,.jpg,.png,.tif}
\usepackage{amssymb, amsmath, pdfpages, amsfonts, calc, times, type1cm, latexsym, xcolor, colortbl, hyperref, bookmark}
\usepackage{mathtools}
\usepackage{bm}
\usepackage{graphicx}
\usepackage{soul}
\usepackage{tabularx}
\usepackage{multirow}
\usepackage{listings}
\usepackage{mathpartir}
\usepackage{xspace}

\usepackage[latin1]{inputenc}
\usepackage[english]{babel}

\usetheme{Szeged}
\usecolortheme{beaver}

\definecolor{websiteGreen}{RGB}{107, 224, 134}
\definecolor{silvery}{RGB}{232, 241, 248}
\definecolor{deepOrange}{RGB}{209, 126, 0}

\definecolor{softTan}{RGB}{240, 240, 223}
\definecolor{deepGreen}{RGB}{87, 149, 115}
\definecolor{lilac}{RGB}{199, 164, 202}
\definecolor{lightOlive}{RGB}{168, 166, 96}
\definecolor{deepOlive}{RGB}{101, 109, 41}

\setbeamercolor*{palette primary}{bg=websiteGreen}
\setbeamercolor*{palette secondary}{bg=white}
\setbeamercolor*{palette tertiary}{bg=silvery}
\setbeamercolor*{palette quaternary}{bg=websiteGreen}

\setbeamercolor{section in head/foot}{fg=deepOrange}
\setbeamerfont{section in head/foot}{series=\bfseries}

\setbeamercolor{titlelike}{parent=palette primary,bg=websiteGreen,fg=white}
\setbeamercolor{frametitle}{parent=palette primary,bg=websiteGreen, fg=white}

\addtobeamertemplate{frametitle}{\vspace*{-0.65\baselineskip}}{}

\newcommand{\tryS}{\textbf{\texttt{try}}\xspace}
\newcommand{\Int}{\textbf{Int}\xspace}
\newcommand{\ok}{\textbf{ok}\xspace}
\newcommand{\catchS}{\textbf{\texttt{catch}}\xspace}

\newcommand{\inl}{\textbf{\texttt{inl}}\xspace}
\newcommand{\inr}{\textbf{\texttt{inr}}\xspace}
\newcommand{\case}{\textbf{\texttt{case}}\xspace}

\newcommand{\bool}{\textbf{Bool}}
\newcommand{\fin}{\textbf{Fin}}

\newtheorem*{conjecture}{Conjecture}
\newtheorem*{proposition}{Proposition}

\title{SIGPLAN}
\subtitle{Type Theory (part $2$ of $n$): Proofs, System $F$}
\date{}

\input{LaTeX/macros.tex}

\begin{document}

\frame{\titlepage}

\begin{frame}{Recap}
    \begin{itemize}
        \item Last time, we introduced type theory, the untyped $\lambda$-calculus, and the simply-typed $\lambda$-calculus
        \item We can show that our languages are type safe/satisfy other nice properties (e.g., programs terminate)
        \item But the simply-typed $\lambda$-calculus was too weak for our purposes
    \end{itemize}
\end{frame}

\begin{frame}{Outline}
    \begin{itemize}
        \item \textbf{Simple-typed $\lambda$-calculus: Recap}
        \item Simple-typed $\lambda$-calculus: Extensions
        \item Algebraic Data Types (briefly)
        \item Curry-Howard Isomorphism (briefly)
        \item Simple-typed $\lambda$-calculus: Proofs
        \item System $F$
    \end{itemize}
\end{frame}

\begin{frame}{Recap: Simply-Typed $\lambda$-Calculus: Syntax, Evaluation}
\begin{tabular}{l r l l}
    $\tau$ & \bnfdef & \texttt{bool} \bnfalt $\tau \to \tau$ \\
    E & \bnfdef & $\true$ \bnfalt $\false$ \bnfalt $x$ \bnfalt $\lambda x : \tau. E$ \bnfalt $E~E$
\end{tabular}

\begin{mathpar}
    \inferrule*[right=Congr]{
        E_1 \to E_1'
    }{ E_1~E_2 \to E_1'~E_2 }

    \inferrule*[right=$\beta$-reduce]{
    }{ (\lambda x : \tau. E) E' \to E[E'/x] }

    \inferrule*[right=$\alpha$-conv.]{
    }{ (\lambda x. E) \to (\lambda y. E[y/x]) }
    \tif \text{$y$ is not in $E$}
\end{mathpar}
\end{frame}

\begin{frame}{Recap: Simply-Typed $\lambda$-Calculus: Type Rules}
\begin{mathpar}
    \inferrule*[right=Bool]{
        b \in \curlys{\true, \false}
    }{ \Gamma \proves b : \texttt{bool} }

    \inferrule*[right=Var]{
    }{ \Gamma, x : \tau \proves x : \tau }

    \inferrule*[right=App]{
        \Gamma \proves f : \tau_1 \to \tau_2
        \\
        \Gamma \proves x : \tau_1
    }{ \Gamma \proves f~x : \tau_2 }

    \inferrule*[right=Abs]{
        \Gamma, x : \tau_1 \proves E : \tau_2
    }{ \Gamma \proves (\lambda x : \tau_1. E) : \tau_1 \to \tau_2 }
\end{mathpar}
\end{frame}

\begin{frame}{Outline}
    \begin{itemize}
        \item Simple-typed $\lambda$-calculus: Recap
        \item \textbf{Simple-typed $\lambda$-calculus: Extensions}
        \item Algebraic Data Types (briefly)
        \item Curry-Howard Isomorphism (briefly)
        \item Simple-typed $\lambda$-calculus: Proofs
        \item System $F$
    \end{itemize}
\end{frame}

\begin{frame}{Extensions of Simply-Typed $\lambda$-calculus}
    \begin{itemize}
        \item We can add many interesting features to the $\lambda$-calculus: we'll do natural numbers, product types, and sum types
        \item We need to extend the syntax
    \end{itemize}

\begin{tabular}{l r l l}
    $\tau$ & \bnfdef & $\ldots$ \bnfalt $\N$ \\
    E & \bnfdef & $\ldots$ \bnfalt $n$ \bnfalt $E + E$
\end{tabular}
\end{frame}

\begin{frame}{Evaluation for Natural Numbers}
    \begin{itemize}
        \item We write $n +_{\N} m$ to mean addition in the ``actual'' natural numbers
        \item You can also think of this as being some hardware implementation of addition: the important thing is it's a single number, not the abstract syntax representation addition of two expressions
    \end{itemize}

    \begin{mathpar}
        \inferrule*[right=Add-Congr-L]{
            E_1 \to E_1'
        }{ E_1 + E_2 \to E_1' + E_2 }

        \inferrule*[right=Add-Congr-R]{
            E_2 \to E_2'
        }{ E_1 + E_2 \to E_1 + E_2' }

        \inferrule*[right=Add]{
        }{ n + m \to n +_{\N} m }
    \end{mathpar}
\end{frame}

\begin{frame}{Typing for Natural Numbers}
    \begin{mathpar}
        \inferrule*[right=T-Nat]{
            n \in \N
        }{ \Gamma \proves n : \N }

        \inferrule*[right=T-Add]{
            \Gamma \proves E_1 : \N
            \\
            \Gamma \proves E_2 : \N
        }{ \Gamma \proves (E_1 + E_2) : \N }
    \end{mathpar}
\end{frame}

\begin{frame}{Extensions of Simply-Typed $\lambda$-calculus: Product types}
    \begin{itemize}
        \item Product types are basically just pairs; records/structs are also product types (but with labels instead of left/right)
    \end{itemize}

\begin{tabular}{l r l l}
    $\tau$ & \bnfdef & $\ldots$ \bnfalt $\tau \times \tau$ \\
    E & \bnfdef & $\ldots$ \bnfalt $(E,E)$ \bnfalt $E_\ell$ \bnfalt $E_r$
\end{tabular}
\end{frame}

\begin{frame}{Evaluation for Product Types}
    \begin{itemize}
        \item Because there's no side-effects in our language, we actually don't need the latter two rules
    \end{itemize}

    \begin{mathpar}
        \inferrule*[right=Pair-Proj-L]{
        }{ (E_1, E_2)_\ell \to E_1 }

        \inferrule*[right=Pair-Proj-R]{
        }{ (E_1, E_2)_r \to E_1 }

        \inferrule*[right=Pair-Congr-L]{
            E_1 \to E_1'
        }{ (E_1, E_2) \to (E_1', E_2) }

        \inferrule*[right=Pair-Congr-R]{
            E_2 \to E_2'
        }{ (E_1, E_2) \to (E_1, E_2') }
    \end{mathpar}
\end{frame}

\begin{frame}{Typing for Product Types}
    \begin{mathpar}
        \inferrule*[right=T-Pair]{
            ??????
        }{ \Gamma \proves (E_1, E_2) : \tau_1 \times \tau_2 }
    \end{mathpar}
\end{frame}

\begin{frame}{Typing for Product Types}
    \begin{mathpar}
        \inferrule*[right=T-Pair]{
            \Gamma \proves E_1 : \tau_1
            \\
            \Gamma \proves E_2 : \tau_2
        }{ \Gamma \proves (E_1, E_2) : \tau_1 \times \tau_2 }

        \inferrule*[right=T-Proj-L]{
            \Gamma \proves E : \tau_1 \times \tau_2
        }{ \Gamma \proves E_\ell : \tau_1 }

        \inferrule*[right=T-Proj-R]{
            \Gamma \proves E : \tau_1 \times \tau_2
        }{ \Gamma \proves E_r : \tau_2 }
    \end{mathpar}
\end{frame}

\begin{frame}[fragile]{Extensions of Simply-Typed $\lambda$-calculus: Sum types}
    \begin{itemize}
        \item Sum types are basically alternatives, like more powerful enums
        \item Saying $E : \tau_1 + \tau_2$ means essentially that either $E$ is of type $\tau_1$ its it's of type $\tau_2$.
        \item The names $\inl$ and $\inr$ are short for ``left injection'' and ``right injection'' ($\inl : \tau_1 \to \tau_1 + \tau_2$ and $\inr : \tau_2 \to \tau_1 + \tau_2$)
    \end{itemize}

\begin{tabular}{l r l l}
    $\tau$ & \bnfdef & $\ldots$ \bnfalt $\tau + \tau$ \\
    E & \bnfdef & $\ldots$ \bnfalt $\inl~E$ \bnfalt $\inr~E$ \bnfalt $\case(E)~\inl~x \mapsto E ; \inr~x \mapsto E$
\end{tabular}
\end{frame}

\begin{frame}{Evaluation for Sum Types}
    \begin{itemize}
        \item Because there's no side-effects in our language, we actually don't need the latter two rules
    \end{itemize}

    \begin{mathpar}
        \inferrule*[right=Case-Congr]{
            E \to E'
        }{ (\case(E)~\inl~x \mapsto E_1 ; \inr~y \mapsto E_2) \\\to (\case(E')~\inl~x \mapsto E_1 ; \inr~y \mapsto E_2) }

        \inferrule*[right=Case-L]{
        }{ (\case(\inl~E)~\inl~x \mapsto E_1 ; \inr~y \mapsto E_2) \to E_1[E/x] }

        \inferrule*[right=Case-R]{
        }{ (\case(\inr~E)~\inl~x \mapsto E_1 ; \inr~y \mapsto E_2) \to E_2[E/y] }
    \end{mathpar}
\end{frame}

\begin{frame}{Typing for Sum Types}
    \begin{itemize}
        \item Something to note here is that these type rules are not deterministic: they can match in many ways for a single term
        \item There's no way to know what the other type in the sum is
        \item We could fix this by adding annotations (e.g., $\inl_{\tau_2}~E : \tau_1 + \tau_2$)
    \end{itemize}

    \begin{mathpar}
        \inferrule*[right=T-In-L]{
            \Gamma \proves E : \tau_1
        }{ \Gamma \proves \inl~E : \tau_1 + \tau_2 }

        \inferrule*[right=T-In-R]{
            \Gamma \proves E : \tau_2
        }{ \Gamma \proves \inr~E : \tau_1 + \tau_2 }

        \inferrule*[right=T-Case]{
            \Gamma \proves E : \tau_1 + \tau_2
            \\
            \Gamma, x : \tau_1 \proves E_1 : \tau
            \\
            \Gamma, y : \tau_2 \proves E_2 : \tau
        }{ \Gamma \proves (\case(E)~\inl~x \mapsto E_1; \inr~y \mapsto E_2) : \tau }
    \end{mathpar}
\end{frame}

\begin{frame}{Outline}
    \begin{itemize}
        \item Simple-typed $\lambda$-calculus: Recap
        \item Simple-typed $\lambda$-calculus: Extensions
        \item \textbf{Algebraic Data Types (briefly)}
        \item Curry-Howard Isomorphism (briefly)
        \item Simple-typed $\lambda$-calculus: Proofs
        \item System $F$
    \end{itemize}
\end{frame}

\begin{frame}[fragile]{Algebraic (?) Data Types}
    \begin{itemize}
        \item Sometimes these types are called ``algebraic types'': this is because in a certain sense, these types form an algebraic structure (a semiring\footnote{also called a rng because it doesn't have \emph{inverses}.})
        \item We consider two types $\tau$ and $\sigma$ to be the same if there is a type-isomorphism: functions $f : \tau \to \sigma$ and $g : \sigma \to \tau$ such that $g \after f = \id{\tau}$ and $f \after g = \id{\sigma}$
\begin{lstlisting}[language=haskell, basicstyle=\small\ttfamily]
f : Bool -> Maybe ()
f True = Just ()
f False = Nothing
g : Maybe () -> Bool
g (Just ()) = True
g Nothing = False
\end{lstlisting}
    \end{itemize}
\end{frame}

\begin{frame}[fragile]{Algebraic Data Types: Examples}
    \begin{itemize}
        \item Define the type $\fin~n = \overbrace{\id{} + \cdots + \id{}}^{n~\text{times}}$.
        \item We have $\bool \times \bool \iso \fin~4$.
        \item We have $\id{} + \id{} \iso \fin~2$ (previous slide)
        \item Exponential types: the type of functions $A \to B$ is called an exponential type (sometimes written $B^A$)
        \item Then $\id{}^{\bool} \iso \id{}$
\begin{lstlisting}[language=haskell, basicstyle=\small\ttfamily]
f : Bool -> ()
f b = ()
g : () -> (Bool -> ())
g () = \b -> ()
\end{lstlisting}
    \end{itemize}
\end{frame}

\begin{frame}[fragile]{Algebraic Data Types: Some Nonsense (sort of)}
    \begin{itemize}
        \item Let's take a minute to think about lists
        \item Lists are kind of like $() + A + A^2 + A^3 + \cdots$ (either empty, one element, two elements, etc.)
        \[
            \texttt{List}~A = \sum_{i=0}^{\infty} A^i
        \]

    \item So then $\texttt{List}~A = \frac{1}{1 - A}$, or $\texttt{List}~A = 1 + A \times (\texttt{List}~A)$
    \item Hmmmm...
\begin{lstlisting}[language=haskell, basicstyle=\small\ttfamily]
data List a = Nil | Cons a (List a)
\end{lstlisting}
    \end{itemize}
\end{frame}

\begin{frame}{Outline}
    \begin{itemize}
        \item Simple-typed $\lambda$-calculus: Recap
        \item Simple-typed $\lambda$-calculus: Extensions
        \item Algebraic Data Types (briefly)
        \item \textbf{Curry-Howard Isomorphism (briefly)}
        \item Simple-typed $\lambda$-calculus: Proofs
        \item System $F$
    \end{itemize}
\end{frame}

\begin{frame}[fragile]{Curry-Howard Isomorphism}
    \begin{itemize}
        \item In addition to viewing these operations as \textbf{algebraic}, we can also view them as \textbf{logical}
        \item Here $A \times B$ means ``$A$ \textbf{and} $B$'', $A + B$ means ``$A$ \textbf{or} $B$'', and $A \to B$ means ``$A$ \textbf{implies} $B$''
        \item So we can encode basic propositional logic in types.
            ``If $A$ implies $B$ and $B$ implies $C$, then $A$ implies $C$''
            \[
                ((A \to B) \times (B \to C)) \to (A \to C)
            \]
        \item Writing a term of some type is equivalent to giving a proof:
            \[
                \lambda p : (A \to B) \times (B \to C). \lambda x : A. p_r~(p_\ell~x)
            \]
        \item Lots more to say about this later (this is basically how proof assistants work)
    \end{itemize}
\end{frame}

\begin{frame}{Outline}
    \begin{itemize}
        \item Simple-typed $\lambda$-calculus: Recap
        \item Simple-typed $\lambda$-calculus: Extensions
        \item Algebraic Data Types (briefly)
        \item Curry-Howard Isomorphism (briefly)
        \item \textbf{Simple-typed $\lambda$-calculus: Proofs}
        \item System $F$
    \end{itemize}
\end{frame}

\begin{frame}{Proving Type Safety}
    \begin{itemize}
        \item Typically we split proving type safety into two parts: \emph{progress} and \emph{preservation}
        \item Informally, type safety means ``well-typed programs don't get stuck.''
    \end{itemize}

\begin{theorem}[Progress]
    Let $E$ be a closed expression (i.e., no free variables).
    If $\Gamma \proves E : \tau$, then either $E$ is a value (i.e, $E = \true$, $E = \false$, or $E = \lambda x. E$), or there is some $E'$ such that $E \to E'$.
\end{theorem}

\begin{theorem}[Preservation]
    If $\Gamma \proves E : \tau$ and $E \to E'$, then $\Gamma \proves E' : \tau$.\footnote{Here we use the same $\tau$ for both; in some languages this may not hold (e.g., because of subtyping)}
\end{theorem}
\end{frame}

\begin{frame}{Recall: Proof Trees}

\begin{minipage}{\dimexpr\textwidth+3em\relax}
\small
\begin{mathpar}
    \inferrule*[right=App]{
        \inferrule*[right=Abs]{
            \inferrule*[right=Bool]{
            }{ x : \texttt{bool} \proves \true : \texttt{bool} }
        }{ \emptyset \proves (\lambda x : \texttt{bool}. \true) : \texttt{bool} \to \texttt{bool} }
        \\
        \inferrule*[right=Bool]{
        }{ \emptyset \proves \false : \texttt{bool} }
    }{ \emptyset \proves (\lambda x : \texttt{bool}. \true)~\false : \texttt{bool} }
\end{mathpar}
\end{minipage}
\end{frame}

\begin{frame}{Empty Slide for Scratch Work}
\end{frame}

\begin{frame}{Proving Type Safety}
\begin{theorem}[Progress]
    Let $E$ be a closed expression (i.e., no free variables).
    If $\Gamma \proves E : \tau$, then either $E$ is a value (i.e, $E = \true$, $E = \false$, or $E = \lambda x. E$), or there is some $E'$ such that $E \to E'$.
\end{theorem}
\begin{proof}
    We proceed by induction on the derivation of $\Gamma \proves E : \tau$.

    \textbf{Case \textsc{Bool}}: In this case, $E \in \curlys{\true, \false}$.
    Then $E$ is a value.

    \textbf{Case \textsc{Var}}: In this case, $E = x$ for some variable $x$.
    Then $E$ is not closed, which is a contradiction.
\end{proof}
\end{frame}

\begin{frame}{Proving Type Safety: Continued}
\begin{proof}
    \textbf{Case \textsc{App}}: In this case, $E = E_1~E_2$ for some expressions $E_1$ and $E_2$.

    Then by inversion of the typing derivation, we have $\Gamma \proves E_1 : \tau_1 \to \tau_2$ and $\Gamma \proves E_2 : \tau_1$.
    Then by the induction hypothesis, we have either $E_1$ is a value, or $E_1 \to E_1'$ for some expression $E_1'$.

    If $E_1$ is a value, then ???? % $E_1 = \lambda x : \tau_1. E'$, and by \textsc{$\beta$-reduce}, we have $(\lambda x : \tau. E')~E_2 \to E'[E_2/x]$. \footnote{technically, we may need to $\alpha$-convert here.}

    If $E_1$ is not a value, then ???? % $E_1 \to E_1'$, and by the congruence rule, $E_1~E_2 \to E_1'~E_2$.

    \textbf{Case \textsc{Abs}}: In this case, $E = \lambda x : \tau. E'$.
    Then ???? % $E$ is a value.

\end{proof}
\end{frame}

\begin{frame}{Proving Type Safety: Recap}
    \begin{itemize}
        \item Why did we do that?
    \end{itemize}
\end{frame}

\begin{frame}{Proving Type Safety: Recap}
    \begin{itemize}
        \item Why did we do that?
        \item So you understand why I keep mentioning reducing the number of cases to check (even worse if we did it with all the product/sum type rules!)
    \end{itemize}
\end{frame}

\begin{frame}{Proving Type Safety: Recap}
    \begin{itemize}
        \item Why did we do that?
        \item \st{So you understand why I keep mentioning reducing the number of cases to check}
        \item To show that our language definition is ``sensible'':
            \begin{itemize}
                \item We didn't forget any evaluation rules
                \item Typechecking guarantees that our programs run successfully
            \end{itemize}
    \end{itemize}
\end{frame}

\begin{frame}{Outline}
    \begin{itemize}
        \item Simple-typed $\lambda$-calculus: Recap
        \item Simple-typed $\lambda$-calculus: Extensions
        \item Algebraic Data Types (briefly)
        \item Curry-Howard Isomorphism (briefly)
        \item Simple-typed $\lambda$-calculus: Proofs
        \item \textbf{System $F$}
    \end{itemize}
\end{frame}

\begin{frame}[fragile]{System $F$}
    \begin{itemize}
        \item Almost Haskell's type system (\textbf{almost} System $F_\omega$ \footnote{As the name suggests, it builds heavily on System $F$})
        \item Basically, it's simply-typed $\lambda$-calculus, but with \emph{parametric polymorphism}
        \item Below is the identity function in Haskell\footnote{You technically need \lstinline[language=haskell, basicstyle=\small\ttfamily]{-XExplicitForAll}}
\begin{lstlisting}[language=haskell, basicstyle=\small\ttfamily]
id : forall a. a -> a
id x = x
\end{lstlisting}
        \item And in System $F$:
        \begin{align*}
            \text{id} &: \forall \alpha. \alpha \to \alpha \\
            \text{id} &= (\Lambda \alpha. \lambda x : \alpha. x)
        \end{align*}
    \end{itemize}
\end{frame}

\begin{frame}[fragile]{System $F$ Syntax}
    \begin{itemize}
        \item System $F$ is again based on the $\lambda$-calculus, but with a much more powerful type system
        \item We've dropped the built-in types altogether, because we don't need them (we also don't need products or sum types)
        \item We use $\alpha$, $\beta$, $\gamma$, etc. for \emph{type variables}
    \end{itemize}
\begin{tabular}{l r l l}
    $\tau$ & \bnfdef & $\tau \to \tau$ \bnfalt $\forall \alpha. \tau$ \\
    E & \bnfdef & $x$ \bnfalt $\lambda x : \tau. E$ \bnfalt $E~E$ \bnfalt $\Lambda \alpha. E$ \bnfalt $E[\tau]$
\end{tabular}
\end{frame}

\begin{frame}[fragile]{System $F$ Evaluation}
    \begin{itemize}
        \item The evaluation rules are the same as before, except we've added one new rule to deal with the new expressions
        \item We overload the substitution syntax to allow substituting for type variables
        \item Technically, we only need to do the substitution so the proof works---we can otherwise erase types at runtime if we cared about efficiency
    \end{itemize}

    \begin{mathpar}
        \inferrule*[right=App-Ty]{
        }{ (\Lambda \alpha. E)[\tau] \to E[\tau/\alpha] }
    \end{mathpar}
\end{frame}

\begin{frame}[fragile]{System $F$ Typechecking}
    \begin{itemize}
        \item We only \textbf{need} two new type rules
        \item However, we add new \emph{type formation} rules for educational purposes (e.g., to disallow $\lambda x : \alpha. x$ without binding $\alpha$)
        \item Extend type contexts to also be able to contain type variables
    \end{itemize}

    \begin{mathpar}
        \inferrule*[right=Var]{
            \Gamma \proves \tau
        }{ \Gamma, x : \tau \proves x : \tau }

        \inferrule*[right=T-Abs]{
            \Gamma, \alpha \proves E : \tau
        }{ \Gamma \proves (\Lambda \alpha. E) : \forall \alpha. \tau }

        \inferrule*[right=T-App]{
            \Gamma \proves E : \forall \alpha. \sigma
        }{ \Gamma \proves E[\tau] : \sigma[\tau / \alpha] }
    \end{mathpar}
\end{frame}

\begin{frame}[fragile]{System $F$ Typechecking: Type Formation}
    \begin{mathpar}
        \inferrule*[right=T-Ok-Var]{
        }{ \Gamma, \alpha \proves \alpha }

        \inferrule*[right=T-Ok-Abs]{
            \Gamma \proves \tau_1
            \\
            \Gamma \proves \tau_2
        }{ \Gamma \proves \tau_1 \to \tau_2 }

        \inferrule*[right=T-Ok-Forall]{
            \Gamma, \alpha \proves \tau
        }{ \Gamma \proves \forall \alpha. \tau }
    \end{mathpar}
\end{frame}

\begin{frame}[fragile]{Comments on System $F$}
    \begin{itemize}
        \item System $F$ actually is in some sense more powerful than default Haskell because we can put $\forall$ anywhere.
            \begin{itemize}
                \item Without extensions, \lstinline[language=haskell, basicstyle=\small\ttfamily]{forall} is only allowed at the ``top'' of types, because it makes type inference possible.
            \end{itemize}
        \item Type inference for System $F$ is undecidable\footnote{without annotations}.
    \end{itemize}
\end{frame}

\begin{frame}[fragile]{Comments on System $F$: Free Theorems}
    \begin{itemize}
        \item You may have heard of free theorems\footnote{If you read the paper you'll see it's not exactly ``free''. See: Philip Wadler. "Theorems for free!" FPCA (1989).}
        \item This refers to the idea that polymorphic types are so strong that just knowing the type of a (closed) term can tell you its bevahior
        \item For example:
            \begin{itemize}
                \item If $f : \forall \alpha. \alpha \to \alpha$, then $f \equiv \Lambda \alpha. \lambda x : \alpha. x$.
                \item If $f : \forall \alpha. \forall \beta. \alpha \times \beta \to \alpha$, $g : A \to B$, and $h : C \to D$ then $g \after f[A][C] = f[A][C] \after (g \times h)$.
\begin{lstlisting}[language=haskell, basicstyle=\small\ttfamily]
fst : (a,c) -> a
g : a -> b
h : c -> d
(x, y) : (a, c)
-- Then
g (fst (x, y)) = fst (g x, h y)
\end{lstlisting}
            \end{itemize}
    \end{itemize}
\end{frame}

\begin{frame}{What Next?}
    \begin{itemize}
        \item Many potential directions we can go:
            \begin{itemize}
                \item Church Encodings (we will talk about this briefly, could be a whole week though)
                \item Featherweight Java
                \item System $F_\omega$
                \item Recursive types
                \item Linear types
                \item Calculus of inductive constructions (Coq); this will require some time
                \item Type theory as foundations for mathematics; will require a lot of time
                \item Pecan
                \item Psamathe's Type System
                \item Something other than type theory :)
            \end{itemize}
        \item What are you interested in?
    \end{itemize}
\end{frame}

\begin{frame}[fragile]{Church Encodings}
    \begin{itemize}
        \item We talked briefly about these before
\begin{align*}
    \texttt{True} & := (\lambda x. \lambda y. x) \\
    \texttt{False} & := (\lambda x. \lambda y. y) \\
    \texttt{And} & := (\lambda b_1. \lambda b_2. b_1~b_2~\texttt{False})
\end{align*}
        \item The basic idea about encoding some type $T$ is to think about how you use values of type $T$, then write a type that reflects that
    \end{itemize}
\end{frame}

\begin{frame}[fragile]{Church Encodings: (Recap) Booleans}
    \begin{itemize}
        \item Encoding \texttt{bool}: how do you use booleans? Mostly as conditions
        \item Abstracting away various kinds of conditions (if statements, while loops, for loops, etc.), we just need if statements and recursion
        \item So the way to use a boolean value \texttt{b} is basically:
\begin{lstlisting}[language=haskell, basicstyle=\small\ttfamily]
if b then val1 else val2
\end{lstlisting}
        \item So if \texttt{b} is true, then we return \texttt{val1}, otherwise we return \texttt{val2}
    \end{itemize}
\begin{align*}
    \texttt{True} & := (\lambda x. \lambda y. x) \\
    \texttt{False} & := (\lambda x. \lambda y. y)
\end{align*}
\end{frame}

\begin{frame}[fragile]{Church Encodings: Booleans in System $F$}
    \begin{itemize}
        \item Previously, we couldn't type these functions (because they take arguments of any types), now we can
        \item We use the same type for both arguments so we know the return type statically
    \end{itemize}
\begin{align*}
    \texttt{True} & : \forall \alpha. \alpha \to \alpha \to \alpha \\
    \texttt{True} & := \Lambda \alpha. \lambda x : \alpha. \lambda y : \alpha. x \\
    \texttt{False} & : \forall \alpha. \alpha \to \alpha \to \alpha \\
    \texttt{False} & := \Lambda \alpha. \lambda x : \alpha. \lambda y : \alpha. y
\end{align*}
\end{frame}

\begin{frame}[fragile]{Church Encodings: Pairs}
    \begin{itemize}
        \item Before, we had to laboriously add new syntax and rules to add pairs
        \item In System $F$, we don't need any of this!
        \item How do you use a pair? Well, you use the two values inside the pair.
        \item In Haskell:
    \end{itemize}
\begin{lstlisting}[language=haskell, basicstyle=\small\ttfamily]
mkPair :: a -> b -> (a -> b -> c) -> c
mkPair a b = \f -> f a b
\end{lstlisting}
\end{frame}

\begin{frame}[fragile]{Church Encodings: Pairs}
    \begin{itemize}
        \item In System $F$ (Do not be afraid):
\begin{align*}
    \texttt{mkPair} &: \forall \alpha. \forall \beta. \alpha \to \beta \to \forall \gamma. (\alpha \to \beta \to \gamma) \to \gamma \\
    \texttt{mkPair} &= \Lambda \alpha. \Lambda \beta. \lambda a : \alpha. \lambda b : \beta. \Lambda \gamma. \lambda f : \alpha \to \beta \to \gamma. f~a~b
\end{align*}
        \item So
\[
    \alpha \times \beta := \forall \gamma. (\alpha \to \beta \to \gamma) \to \gamma
\]
        \item Using this abbreviation (which we can't formalize without higher-kinded types)
\[
    \texttt{mkPair} : \forall \alpha. \forall \beta. \alpha \to \beta \to \alpha \times \beta
\]
    \end{itemize}
\begin{lstlisting}[language=haskell, basicstyle=\small\ttfamily]
mkPair :: a -> b -> (a -> b -> c) -> c
mkPair a b = \f -> f a b
\end{lstlisting}
\end{frame}

\begin{frame}{Linear Typing}
    \begin{itemize}
        \item TODO?
    \end{itemize}
\end{frame}

\begin{frame}{Featherweight Java?}
    \begin{itemize}
        \item TODO?
    \end{itemize}
\end{frame}

\begin{frame}{Psamathe Type System}
    \begin{itemize}
        \item TODO?
    \end{itemize}
\end{frame}

\begin{frame}{Linear Typing}
    \begin{itemize}
        \item TODO?
    \end{itemize}
\end{frame}

\end{document}

